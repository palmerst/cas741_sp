\documentclass[12pt]{article}

\usepackage{amsmath, mathtools}
\usepackage{amsfonts}
\usepackage{amssymb}
\usepackage{graphicx}
\usepackage{colortbl}
\usepackage{xr}
\usepackage{hyperref}
\usepackage{longtable}
\usepackage{xfrac}
\usepackage{tabularx}
\usepackage{float}
\usepackage{siunitx}
\usepackage{booktabs}
\usepackage{caption}
\usepackage{pdflscape}
\usepackage{afterpage}
\usepackage{tabu}
\usepackage{verbatim}
\usepackage[round]{natbib}
\usepackage{url}

\makeatletter
\newcommand*\bigcdot{\mathpalette\bigcdot@{.7}}
\newcommand*\bigcdot@[2]{\mathbin{\vcenter{\hbox{\scalebox{#2}{$\m@th#1\bullet$}}}}}
\makeatother

%\usepackage{refcheck}

\hypersetup{
    bookmarks=true,         % show bookmarks bar?
      colorlinks=true,       % false: boxed links; true: colored links
    linkcolor=red,          % color of internal links (change box color with linkbordercolor)
    citecolor=green,        % color of links to bibliography
    filecolor=magenta,      % color of file links
    urlcolor=cyan           % color of external links
}

\input{../Comments}

\newcommand{\sskip}{\vskip 1mm}

% For easy change of table widths
\newcommand{\colZwidth}{1.0\textwidth}
\newcommand{\colAwidth}{0.13\textwidth}
\newcommand{\colBwidth}{0.82\textwidth}
\newcommand{\colCwidth}{0.1\textwidth}
\newcommand{\colDwidth}{0.05\textwidth}
\newcommand{\colEwidth}{0.8\textwidth}
\newcommand{\colFwidth}{0.17\textwidth}
\newcommand{\colGwidth}{0.5\textwidth}
\newcommand{\colHwidth}{0.28\textwidth}

% Used so that cross-references have a meaningful prefix
\newcounter{defnum} %Definition Number
\newcommand{\dthedefnum}{GD\thedefnum}
\newcommand{\dref}[1]{GD\ref{#1}}
\newcounter{datadefnum} %Datadefinition Number
\newcommand{\ddthedatadefnum}{DD\thedatadefnum}
\newcommand{\ddref}[1]{DD\ref{#1}}
\newcounter{theorynum} %Theory Number
\newcommand{\tthetheorynum}{T\thetheorynum}
\newcommand{\tref}[1]{T\ref{#1}}
\newcounter{tablenum} %Table Number
\newcommand{\tbthetablenum}{T\thetablenum}
\newcommand{\tbref}[1]{TB\ref{#1}}
\newcounter{assumpnum} %Assumption Number
\newcommand{\atheassumpnum}{P\theassumpnum}
\newcommand{\aref}[1]{A\ref{#1}}
\newcounter{goalnum} %Goal Number
\newcommand{\gthegoalnum}{P\thegoalnum}
\newcommand{\gsref}[1]{GS\ref{#1}}
\newcounter{instnum} %Instance Number
\newcommand{\itheinstnum}{IM\theinstnum}
\newcommand{\iref}[1]{IM\ref{#1}}
\newcounter{reqnum} %Requirement Number
\newcommand{\rthereqnum}{P\thereqnum}
\newcommand{\rref}[1]{R\ref{#1}}
\newcounter{lcnum} %Likely change number
\newcommand{\lthelcnum}{LC\thelcnum}
\newcommand{\lcref}[1]{LC\ref{#1}}
\newcommand{\sref}[1]{\S~\ref{#1}}

\newcommand{\progname}{SpecGen} % PUT YOUR PROGRAM NAME HERE

\usepackage{fullpage}

\begin{document}
\pagenumbering{gobble}

\title{CAS 741: SRS\\[10pt]\Large Aqueous Speciation Diagram Generator}
\author{Steven Palmer\\\texttt{palmes4}}
\date{\today}
	
\maketitle

~\newpage

\pagenumbering{roman}

\setcounter{secnumdepth}{0}

\section{Revision History}

\begin{table}[hp]
\caption{Revision History} \label{TblRevisionHistory}
\begin{tabularx}{\textwidth}{llX}
\toprule
\textbf{Date} & \textbf{Developer(s)} & \textbf{Change}\\
\midrule
10.5.2017 & S. Palmer & First revision of document\\
\bottomrule
\end{tabularx}
\end{table}

~\newpage

\section{Reference Material}

This section records information for easy reference.

\subsection{Table of Units}

Throughout this document SI (Syst\`{e}me International d'Unit\'{e}s) is employed
as the unit system.  In addition to the basic units, several derived units are
used as described below.  For each unit, the symbol is given followed by a
description of the unit and the SI name.
~\newline

\renewcommand{\arraystretch}{1.2}
%\begin{table}[ht]
  \noindent \begin{tabular}{l l l} 
    \toprule		
    \textbf{symbol} & \textbf{unit} & \textbf{SI}\\
    \midrule 
    \si{\metre} & length & metre\\
    \si{\kilogram} & mass & kilogram\\
    \si{\second} & time & second\\
    \si{\celsius} & temperature & centigrade\\
    \si{\mole} & amount of substance & mole\\
    \si{\litre} & volume & litre ($\si{\litre} = 10^{-3}\cdot\si{\cubic\metre}$)\\
    \si{\pascal} & pressure & pascal ($\si{\pascal} = \si{\kilogram\per\metre\per\square\second}$)\\
    \si{\bar} & pressure & bar ($\si{\bar} = 10^{5}\cdot\si{\pascal}$)\\
    \bottomrule
  \end{tabular}
  %	\caption{Provide a caption}
%\end{table}

\subsection{Table of Symbols}

The table that follows summarizes the symbols used in this document along with
their units.  The choice of symbols was made to be consistent with chemistry literature.  The symbols are listed in alphabetical order.

\renewcommand{\arraystretch}{1.2}
%\noindent \begin{tabularx}{1.0\textwidth}{l l X}
\noindent \begin{longtable*}{l l p{12cm}} \toprule
\textbf{symbol} & \textbf{unit} & \textbf{description}\\
\midrule 
$A$ & -- & reactant species in a chemical equation
\\
$B$ & -- & product species in a chemical equation
\\ 
$K$ & -- & equilibrium constant
\\ 
$p$ & -- & the total number of user supplied chemical equilibrium reactions
\\ 
$\overline {RK}_{in}$ & -- & the input set of all pairs of reactions and associated equilibrium constants 
\\ 
$\overline S$ & -- & a set of aqueous solute species in a chemical system
\\ 
$\overline S_{in}$ & -- & the input set of all aqueous solute species in the chemical system
\\ 
$\overline T_{in}$ & -- & the input set of total concentrations of chemical elements
\\ 
$X$ & -- & arbitrary chemical species
\\
$x$ & -- & arbitrary chemical element
\\
$x_{tot}$ & \si[per-mode=symbol] {\mole\per\litre} & total concentration of an arbitrary chemical element $x$
\\
$[X]$ & \si[per-mode=symbol] {\mole\per\litre} & concentration of the arbitrary chemical species $X$
\\
$\{X\}$ & -- & activity of the arbitrary chemical species $X$
\\
$z_{X}$ & -- & charge number of an arbitrary aqueous solute $X$
\\
\bottomrule
\end{longtable*}


\subsection{Abbreviations and Acronyms}

\renewcommand{\arraystretch}{1.2}
\begin{tabular}{l l} 
  \toprule		
  \textbf{symbol} & \textbf{description}\\
  \midrule 
  A & Assumption\\
  DD & Data Definition\\
  GD & General Definition\\
  GS & Goal Statement\\
  IM & Instance Model\\
  LC & Likely Change\\
  PS & Physical System Description\\
  R & Requirement\\
  SATP & Standard Ambient Temperature and Pressure\\
  SRS & Software Requirements Specification\\
  \progname{} & The Aqueous Speciation Diagram Generator program\\
  T & Theoretical Model\\
  \bottomrule
\end{tabular}\\

\newpage

\tableofcontents

~\newpage

\pagenumbering{arabic}

\setcounter{secnumdepth}{3}

\section{Introduction}

\subsection{Purpose of Document}
The purpose of this document is to describe the requirements for the Aqueous Speciation Diagram Generator program (herein referred to as \progname), a software product that will produce a speciation diagram given a set of chemical reactions and element totals that define a chemical system as inputs.  The goals and theoretical models used in the SpecGen code are provided, with an emphasis on explicitly identifying assumptions and unambiguous definitions. This document is intended to be used as a reference to provide all information necessary to understand and verify the transformation of inputs to outputs. The SRS is abstract: the contents describe the problem being solved, but not how to solve it.

This document will be used as a starting point for subsequent development phases, including writing the design specification and the software verification and validation plan. The design document will show how the requirements are to be realized, including decisions on the numerical algorithms and programming environment. The verification and validation plan will show the steps that will be used to increase confidence in the software documentation and the implementation. Although the SRS fits in a series of documents that follow the so-called waterfall model, the actual development process is not constrained in any way. Even when the waterfall model is not followed, as Parnas and Clements point out, the most logical way to present the documentation is still to "fake" a rational design process.

\subsection{Scope of Requirements} 
The scope of the requirements includes collecting all input parameters that describe a chemical system. Given the appropriate inputs, the code for SpecGen is intended to use the data to produce a speciation diagram in the pH range 0 to 14. 

\subsection{Characteristics of Intended Reader} 
Reviewers of this documentation should have an understanding of first year undergraduate physical chemistry.  Knowledge about chemical activity is assumed.


\subsection{Organization of Document}
The organization of this document follows the template for an SRS for scientific computing software proposed by [1] and [2] (in Sec:Refe), with some aspects taken from [3]. The presentation follows the standard pattern of presenting goals, theories, definitions, and assumptions. For readers that would like a more bottom up approach, they can start reading the data definitions in Sec:DataDefi and trace back to find any additional information they require.

The goal statements are refined to the theoretical models, and the theoretical models to the instance models. The data definitions are used to support the definitions of the different models.

\section{General System Description}
This section identifies the interfaces between the system and its environment,
describes the user characteristics and lists the system constraints.

\subsection{System Context}

\begin{itemize}
\item User Responsibilities:
\begin{itemize}
\item Ensure that the input data is correct.
\item Ensure that the input data is sufficient to specify a chemical system that has a unique speciation solution.
\end{itemize}
\item \progname{} Responsibilities:
\begin{itemize}
\item Detect data type mismatch, such as a string of characters instead of a
  floating point number.
\item Determine if the inputs satisfy the required physical and software constraints.
\item Solve the system of equations arising from the input data to generate the output data.
\item Generate a plot of the output data.
\end{itemize}
\end{itemize}

\subsection{User Characteristics} \label{SecUserCharacteristics}
The end user of \progname{} should have an understanding of first year undergraduate physical chemistry.

\subsection{System Constraints}
There are no system constraints.

\section{Specific System Description}

This section first presents the problem description, which gives a high-level
view of the problem to be solved.  This is followed by the solution characteristics
specification, which presents the assumptions, theories, definitions and finally
the instance models.

\subsection{Problem Description} \label{Sec_pd}
Chemical speciation refers to the stable (equilibrium) distribution of chemical species in a given chemical system.   Speciation diagrams, which plot species concentrations against an independently varied parameter of the system, are useful tools for displaying speciation data in a concise and easy to use format.  The production of speciation diagrams requires solving a set of non-linear equations which arise from the chemical reactions of the species present in a chemical system.  As the number of reactions taking place in a system increases, producing speciation diagrams can quickly become a tedious undertaking when done manually.

\progname{} will produce a speciation diagram given a set of chemical reactions, equilibrium constants, and element totals that define a chemical system.  \progname{} will be specific to speciation of ions in aqueous systems under varying pH, which is of particular importance in the fields of aqueous process engineering and hydrometallurgy.  The diagram generated by \progname{} will plot speciation of all aqueous species (excluding H$^+$ and OH$^-$) across the pH range 0 to 14.

\subsubsection{Terminology and Definitions}

This subsection provides a list of terms that are used in the subsequent
sections and their meaning, with the purpose of reducing ambiguity and making it
easier to correctly understand the requirements:

\begin{description}
\item[Activity:] the ``effective concentration'' of a species in a chemical system
\item[Equilibrium:] the state of a chemical system in which species concentrations are stable
\item[SATP:] standard ambient temperature pressure $(T = 25^{\circ}\textrm{C}, ~P = 1~\textrm{bar})$
\end{description}

\subsubsection{Physical System Description}

The physical system of \progname{} includes the following elements:

\begin{itemize}

\item[PS1:] An unspecified volume of liquid water containing dissolved chemical species under equilibrium conditions.

\end{itemize}


\subsubsection{Goal Statements}

\noindent Given a set of chemical species present in an aqueous chemical system, a set of reversible reactions and associated equilibrium constants that the species are subject to, and the total concentrations of all elements present in the species set (excluding O and H), the goal statements are:

\begin{itemize}

\item[GS\refstepcounter{goalnum}\thegoalnum \label{G_solve}:] Calculate the equilibrium concentrations of all chemical species present in the system over the pH range 0 to 14.
\item[GS\refstepcounter{goalnum}\thegoalnum \label{G_plot}:] Generate a plot of the equilibrium concentrations of all chemical species present in the system versus pH.
\end{itemize}

\subsection{Solution Characteristics Specification}

The instance models that govern \progname{} are presented in
\sref{sec_instance}.  The information to understand the meaning of the
instance models and their derivation is also presented, so that the instance
models can be verified.

\subsubsection{Assumptions}

This section simplifies the original problem and helps in developing the
theoretical model by filling in the missing information for the physical
system. The numbers given in the square brackets refer to the theoretical model
[T], general definition [GD], data definition [DD], instance model [IM], or
likely change [LC], in which the respective assumption is used.

\begin{itemize}

\item[A\refstepcounter{assumpnum}\theassumpnum \label{A_SATP}:]
  All generated speciation diagrams are at SATP conditions.
\item[A\refstepcounter{assumpnum}\theassumpnum \label{A_K_SATP}:]
  In accordance with \aref{A_SATP}, all user supplied equilibrium constants are assumed to be values measured under SATP conditions.
\item[A\refstepcounter{assumpnum}\theassumpnum \label{A_only_wat}:]
  The chemical system is assumed be a single-solvent solution, where the solvent is liquid water at SATP.
\item[A\refstepcounter{assumpnum}\theassumpnum \label{A_only_aq}:]
  The chemical species in all of the user supplied equations are assumed to be aqueous solutes.
\item[A\refstepcounter{assumpnum}\theassumpnum \label{A_act_wat}:]
  The activity of liquid water is assumed to be 1.0.
\item[A\refstepcounter{assumpnum}\theassumpnum \label{A_act_aq}:]
  The activities of all aqueous solute species are assumed to be equal to their concentration.
\item[A\refstepcounter{assumpnum}\theassumpnum \label{A_rev}:]
  All chemical equations supplied by the user are assumed to be reversible (finite equilibrium constant).
\end{itemize}

\subsubsection{Theoretical Models}\label{sec_theoretical}

This section focuses on the general equations and laws that \progname{} is based
on.

~\newline

\noindent
\begin{minipage}{\textwidth}
\renewcommand*{\arraystretch}{1.5}
\tabulinesep=1.5mm
\begin{tabu}{| p{\colAwidth} | p{\colBwidth}|}
  \hline
  \rowcolor[gray]{0.9}
  Number& T\refstepcounter{theorynum}\thetheorynum \label{T_RREQ}\\
  \hline
  Label&\bf Equational representation of reversible chemical reactions\\
  \hline
  Equation&  ${\alpha}_1 {A}_{1} + \cdots + {\alpha}_n {A}_{n} \rightleftharpoons {\beta}_1 {B}_{1} + \cdots + {\beta}_m {B}_{m}$\\
  \hline
  Description & 
                The above equation represents a reversible chemical reaction where:\\
              & $n$ is the number of reactants\\
              & $m$ is the number of products\\
              & $A_i$ is the $i^{\textrm{th}}$ reactant species\\
              & $\alpha_i$ is the stoichiometric coefficient of the $i^{\textrm{th}}$ reactant\\
              & $B_j$ is the $j^{\textrm{th}}$ product species\\
              & $\beta_j$ is the stoichiometric coefficient of the $j^{\textrm{th}}$ product\\
  \hline
  Source &~\cite{wiki:rr}\\
  \hline
  Ref.\ By & \dref{ROCT}\\
  \hline
\end{tabu}
\end{minipage}\\
~\newline
~\newline
\noindent
\begin{minipage}{\textwidth}
\renewcommand*{\arraystretch}{1.5}
\tabulinesep=1.5mm
\begin{tabu}{| p{\colAwidth} | p{\colBwidth}|}
  \hline
  \rowcolor[gray]{0.9}
  Number& T\refstepcounter{theorynum}\thetheorynum \label{T_MB}\\
  \hline
  Label&\bf Conservation of mass (concentration balance)\\
  \hline
  Equation&  $x_{tot} = \sum\{X : S ~|~ x \in X ~\bigcdot~ \#x\langle X\rangle \cdot [X]\}$\\
  \hline
  Description & 
                The total dissolved concentration of an element in an aqueous solution is equal to the sum of the concentrations of all species which contain that element.  This is captured in the above equation, where:\\
              & $x$ is a chemical element\\
              & $x_{tot}$ is the total dissolved concentration of $x$ in a solution\\
              & $X$ is an aqueous species in solution\\
              & $S$ is a set of all aqueous species in a solution\\
              & $\#x\langle S\rangle$ is the number of occurrences of $x$ in $X$\\
              & $[X]$ is the concentration of $X$ in solution\\
  \hline
  Source &\\
  \hline
  Ref.\ By & \dref{ROCT}\\
  \hline
\end{tabu}
\end{minipage}\\
~\newline
\noindent
\begin{minipage}{\textwidth}
\renewcommand*{\arraystretch}{1.5}
\tabulinesep=1.5mm
\begin{tabu}{| p{\colAwidth} | p{\colBwidth}|}
  \hline
  \rowcolor[gray]{0.9}
  Number& T\refstepcounter{theorynum}\thetheorynum \label{T_CB}\\
  \hline
  Label&\bf Charge neutrality of aqueous solutions\\
  \hline
  Equation&  $\sum\{X : S ~\bigcdot~ z_{X} \cdot [X]\} = 0$\\
  \hline
  Description & 
                Aqueous solutions must be electrically neutral, and thus the concentration of positive charge from dissolved solutes must balance with the concentration of negative charge from dissolved solutes.  This is reflected in the equation above, where:\\
              & $X$ is an aqueous species in solution\\
              & $S$ is a set of all aqueous species a solution\\
              & $z_{X}$ is the charge number of $X$\\
              & $[X]$ is the concentration of $X$\\
  \hline
  Source &\\
  \hline
  Ref.\ By & \dref{ROCT}\\
  \hline
\end{tabu}
\end{minipage}\\

~\newline
\noindent
\begin{minipage}{\textwidth}
\renewcommand*{\arraystretch}{1.5}
\tabulinesep=1.5mm
\begin{tabu}{| p{\colAwidth} | p{\colBwidth}|}
\hline
\rowcolor[gray]{0.9}
Number& T\refstepcounter{theorynum}\thetheorynum \label{T_LMAF}\\
\hline
Label &\bf Law of mass action (forward) \\
\hline
Equation&$ r_+ = k_+ \displaystyle\prod_{i=1}^{n} \{A_i\}^{\alpha_i} $  \\
\hline
Description &
The law of mass action describes the forward and backward reaction rates for a chemical reaction.  For a chemical reaction expressed in the equational form given in \tref{T_RREQ}, the equation above expresses the law of mass action in the forward direction, where:\\

& $r_+$ is the forward reaction rate (\si{\mole\per\second}).\\
& $k_+$ is the forward rate constant (\si{\mole\per\second}).\\
& $n$ is the number of reactants.\\
& $A_i$ is the $i^{th}$ reactant species.\\
& $\{A_i\}$ is the activity of $A_i$.\\
& $\alpha_i$ is the stoichiometric coefficient of the $i^{\textrm{th}}$ reactant.\\
\hline
  Source &~\cite{wiki:eq}\\
  \hline
  Ref.\ By & \ddref{FluxCoil}, \ddref{FluxPCM}\\
  \hline
\end{tabu}
\end{minipage}\\
~\newline

\noindent
\begin{minipage}{\textwidth}
\renewcommand*{\arraystretch}{1.5}
\tabulinesep=1.5mm
\begin{tabu}{| p{\colAwidth} | p{\colBwidth}|}
\hline
\rowcolor[gray]{0.9}
Number& T\refstepcounter{theorynum}\thetheorynum \label{T_LMAB}\\
\hline
Label &\bf Law of mass action (backward) \\
\hline
Equation&$ r_- = k_- \displaystyle\prod_{j=1}^{m} \{B_j\}^{\beta_j} $ \\
\hline
Description &
The law of mass action describes the forward and backward reaction rates for a chemical reaction.  For a chemical reaction expressed in the equational form given in \tref{T_RREQ}, the equation above expresses the law of mass action in the backward direction, where:\\

& $r_-$ is the backward reaction rate (\si{\mole\per\second}).\\
& $k_-$ is the backward rate constant (\si{\mole\per\second}).\\
& $m$ is the number of products.\\
& $B_j$ is the $j^{th}$ product species.\\
& $\{B_j\}$ is the activity of $B_j$.\\
& $\beta_j$ is the stoichiometric coefficient of the $j^{\textrm{th}}$ product.\\
\hline
  Source &~\cite{wiki:eq}\\
  \hline
  Ref.\ By & \ddref{FluxCoil}, \ddref{FluxPCM}\\
  \hline
\end{tabu}
\end{minipage}\\
~\newline

\noindent
\begin{minipage}{\textwidth}
\renewcommand*{\arraystretch}{1.5}
\tabulinesep=1.5mm
\begin{tabu}{| p{\colAwidth} | p{\colBwidth}|}
\hline
\rowcolor[gray]{0.9}
Number& T\refstepcounter{theorynum}\thetheorynum \label{T_EQ}\\
\hline
Label &\bf Chemical equilibrium \\
\hline
Equation&$ k_+ \displaystyle\prod_{i=1}^{n} \{A_i\}^{\alpha_i} = k_- \displaystyle\prod_{j=1}^{m} \{B_j\}^{\beta_j} $ \\
\hline
Description &
When a chemical reaction is in equilibrium, the forward and reverse reaction rates are the same.  Using \tref{T_LMAF} and \tref{T_LMAB}, this idea is expressed in the equation above, where:\\
& $k_+$ is the forward rate constant (\si{\mole\per\second}).\\
& $n$ is the number of reactants.\\
& $A_i$ is the $i^{th}$ reactant species.\\
& $\{A_i\}$ is the activity of $A_i$.\\
& $\alpha_i$ is the stoichiometric coefficient of the $i^{\textrm{th}}$ reactant.\\
& $k_-$ is the backward rate constant (\si{\mole\per\second}).\\
& $m$ is the number of products.\\
& $B_j$ is the $j^{th}$ product species.\\
& $\{B_j\}$ is the activity of $B_j$.\\
& $\beta_j$ is the stoichiometric coefficient of the $j^{\textrm{th}}$ product.\\
\hline
  Source &~\cite{wiki:eq}\\
  \hline
  Ref.\ By & \ddref{FluxCoil}, \ddref{FluxPCM}\\
  \hline
\end{tabu}
\end{minipage}\\

~\newline

\noindent
\begin{minipage}{\textwidth}
\renewcommand*{\arraystretch}{1.5}
\tabulinesep=1.5mm
\begin{tabu}{| p{\colAwidth} | p{\colBwidth}|}
\hline
\rowcolor[gray]{0.9}
Number& T\refstepcounter{theorynum}\thetheorynum \label{T_K}\\
\hline
Label& \bf Equilibrium constant\\
\hline
% Units& $Mt^{-3}$\\
% \hline
  Equation&$K = \dfrac{r_+}{r_-} = \dfrac{\prod_{j=1}^{m} \{B_j\}^{\beta_j}}{\prod_{i=1}^{n} \{A_i\}^{\alpha_i}}$\\
  \hline
  Description & 
                The ratio of the forward and backward rate constants for a reaction forms a new constant called the equilibrium constant.  Using \tref{T_EQ}, we can relate the equilibrium constant to the activities of the products and reactants.  This is shown in the equation above, where:\\
& $K$ is the equilibrium constant.\\
& $r_+$ is the forward reaction rate (\si{\mole\per\second}).\\
& $k_+$ is the forward rate constant (\si{\mole\per\second}).\\
& $n$ is the number of reactants.\\
& $A_i$ is the $i^{th}$ reactant species.\\
& $\{A_i\}$ is the activity of $A_i$.\\
& $\alpha_i$ is the stoichiometric coefficient of the $i^{\textrm{th}}$ reactant.\\
& $r_-$ is the backward reaction rate (\si{\mole\per\second}).\\
& $k_-$ is the backward rate constant (\si{\mole\per\second}).\\
& $m$ is the number of products.\\
& $B_j$ is the $j^{th}$ product species.\\
& $\{B_j\}$ is the activity of $B_j$.\\
& $\beta_j$ is the stoichiometric coefficient of the $j^{\textrm{th}}$ product.\\
  \hline
  Sources&~\cite{wiki:eq}  \\
  \hline
  Ref.\ By & \iref{ewat}\\
  \hline
\end{tabu}
\end{minipage}\\

\subsubsection{General Definitions}\label{sec_gendef}

This section collects the laws and equations that will be used in deriving the
data definitions, which in turn are used to build the instance models.
\wss{Some projects may not have any content for this section, but the section
  heading should be kept.}  \wss{Modify the examples below for your problem, and
  add additional definitions as appropriate.}

~\newline

\noindent
\begin{minipage}{\textwidth}
\renewcommand*{\arraystretch}{1.5}
\tabulinesep=1.5mm
\begin{tabu}{| p{\colAwidth} | p{\colBwidth}|}
\hline
\rowcolor[gray]{0.9}
Number& GD\refstepcounter{defnum}\thedefnum \label{NL}\\
\hline
Label &\bf Water dissociation equilibrium \\
\hline
% Units&$MLt^{-3}T^0$\\
% \hline
SI Units&--\\
\hline
Equation&$ K_{H_2O}:~10^{-14} = \{H^+\} \cdot \{OH^-\} $  \\
\hline
Description &
This is the equilibrium equation for the dissociation of water.  The equation arises
from \tref{T_K} using the value of $10^{-14}$ for the equilibrium constant at SATP, where:\\
&$\{H^+\}$ is the activity of $H^+$\\
&$\{OH^-\}$ is the activity of $OH^-$\\
\hline
  Source &~\cite[p.\ 8]{Incropera2007}\\
  \hline
  Ref.\ By & \ddref{FluxCoil}, \ddref{FluxPCM}\\
  \hline
\end{tabu}
\end{minipage}\\


\subsubsection{Data Definitions}\label{sec_datadef}

This section collects and defines all the data needed to build the instance
models. The dimension of each quantity is also given.  \wss{Modify the examples
  below for your problem, and add additional definitions as appropriate.}

~\newline

\noindent
\begin{minipage}{\textwidth}
\renewcommand*{\arraystretch}{1.5}
\tabulinesep=1.5mm
\begin{tabu}{| p{\colAwidth} | p{\colBwidth}|}
\hline
\rowcolor[gray]{0.9}
Number& DD\refstepcounter{datadefnum}\thedatadefnum \label{FluxCoil}\\
\hline
Label& \bf Set of equilibria\\
\hline
Symbol &$\overline K$\\
\hline
% Units& $Mt^{-3}$\\
% \hline
  SI Units & --\\
  \hline
  Equation&$\overline K = \left\{(R,K_R):\overline{RK}_{in} ~\bigcdot~ K_R = \dfrac{\prod_{j=1}^{m} \{B_{j}\}^{\beta_{j}}}{\prod_{i=1}^{n} \{A_{i}\}^{\alpha_{i}}}~\textrm{with reaction}~R\right\} \cup K_{H_2O}$\\
  \hline
  Description & Uses \tref{T_K}
  \\
  \hline
  Sources&~\cite{wiki:eq}  \\
  \hline
  Ref.\ By & \iref{ewat}\\
  \hline
\end{tabu}
\end{minipage}\\

~\newline

\noindent
\begin{minipage}{\textwidth}
\renewcommand*{\arraystretch}{1.5}
\tabulinesep=1.5mm
\begin{tabu}{| p{\colAwidth} | p{\colBwidth}|}
\hline
\rowcolor[gray]{0.9}
Number& DD\refstepcounter{datadefnum}\thedatadefnum \label{FluxCoil}\\
\hline
Label& \bf Set of mass balance equations\\
\hline
Symbol &$\overline M$\\
\hline
% Units& $Mt^{-3}$\\
% \hline
  SI Units & --\\
  \hline
  Equation&$\overline M = \left\{x_{tot}:\overline T_{in} ~\bigcdot~ \sum\{X : \overline S_{in} ~|~ x \in X ~\bigcdot~ \#x\langle X\rangle \cdot [X]\} = x_{tot}\right\}$\\
  \hline
  Description & Uses \tref{T_MB}
  \\
  \hline
  Sources&~\cite{wiki:eq}  \\
  \hline
  Ref.\ By & \iref{ewat}\\
  \hline
\end{tabu}
\end{minipage}\\

~\newline

\noindent
\begin{minipage}{\textwidth}
\renewcommand*{\arraystretch}{1.5}
\tabulinesep=1.5mm
\begin{tabu}{| p{\colAwidth} | p{\colBwidth}|}
\hline
\rowcolor[gray]{0.9}
Number& DD\refstepcounter{datadefnum}\thedatadefnum \label{FluxCoil}\\
\hline
Label& \bf Charge balance equation\\
\hline
Symbol &$C$\\
\hline
% Units& $Mt^{-3}$\\
% \hline
  SI Units & --\\
  \hline
  Equation&$C:~ \sum\{X : \overline S_{in} ~\bigcdot~ z_{X} \cdot [X]\} = 0$\\
  \hline
  Description & Uses \tref{T_K}
  \\
  \hline
  Sources&~\cite{wiki:eq}  \\
  \hline
  Ref.\ By & \iref{ewat}\\
  \hline
\end{tabu}
\end{minipage}\\

\subsubsection{Instance Models} \label{sec_instance}    

This section transforms the problem defined in \sref{Sec_pd} into 
one which is expressed in mathematical terms. It uses concrete symbols defined 
in \sref{sec_datadef} to replace the abstract symbols in the models 
identified in \sref{sec_theoretical} and \sref{sec_gendef}.

The goals \wss{reference your goals} are solved by \wss{reference your instance
  models}.  \wss{other details, with cross-references where appropriate.}
\wss{Modify the examples below for your problem, and add additional models as
  appropriate.}

~\newline

%Instance Model 1

\noindent
\begin{minipage}{\textwidth}
\renewcommand*{\arraystretch}{1.5}
\tabulinesep=1.5mm
\begin{tabu}{| p{\colAwidth} | p{\colBwidth}|}
  \hline
  \rowcolor[gray]{0.9}
  Number& IM\refstepcounter{instnum}\theinstnum \label{ewat}\\
  \hline
  Label& \bf System of non-linear equations to find equilibrium concentrations\\
  \hline
  Input&$\overline K$, $\overline M$, $C$\\
  \hline
  Output&$\{X:S ~|~ X \neq OH^- \land X \neq H^+ ~\bigcdot~ \{[X]~\textrm{for}~0 \leq pH \leq 14\} \}$,\\
  &when the system of non-linear equations $\overline K \cup \overline M \cup C$ is satisfied\\
  \hline
  Description&
  \\
  \hline
  Sources&~\cite{Lightstone2012} \ \\
  \hline
  Ref.\ By & \iref{epcm}\\
  \hline
\end{tabu}
\end{minipage}\\

%~\newline

\subsubsection*{Derivation of ...}

\wss{May be necessary to include this subsection in some cases.}

\subsubsection{Data Constraints} \label{sec_DataConstraints}    

Tables~\ref{TblInputVar} and \ref{TblOutputVar} show the data constraints on the
input and output variables, respectively.  The column for physical constraints gives
the physical limitations on the range of values that can be taken by the
variable.  The column for software constraints restricts the range of inputs to
reasonable values.  The constraints are conservative, to give the user of the
model the flexibility to experiment with unusual situations.  The column of
typical values is intended to provide a feel for a common scenario.  The
uncertainty column provides an estimate of the confidence with which the
physical quantities can be measured.  This information would be part of the
input if one were performing an uncertainty quantification exercise.

The specification parameters in Table~\ref{TblInputVar} are listed in
Table~\ref{TblSpecParams}.

\begin{table}[!h]
  \caption{Input Variables} \label{TblInputVar}
  \renewcommand{\arraystretch}{1.2}
\noindent \begin{longtable*}{l l l l c} 
  \toprule
  \textbf{Var} & \textbf{Physical Constraints} & \textbf{Software Constraints} &
                             \textbf{Typical Value} & \textbf{Uncertainty}\\
  \midrule 
  $x_{tot}$ & $L > 0$ & $x_{{tot}_{\text{min}}} \leq x_{tot} $ & 0.1 \si[per-mode=symbol] {\mole\per\litre} & 2.5\%
  \\
  \bottomrule
  \multicolumn{5}{l}{\scriptsize * The user.}\\
  \multicolumn{5}{l}{\scriptsize ** There must be an accompanying equilibrium constant for every equation supplied by the user.}
\end{longtable*}
\end{table}


\begin{table}[!h]
\caption{Specification Parameter Values} \label{TblSpecParams}
\renewcommand{\arraystretch}{1.2}
\noindent \begin{longtable*}{l l} 
  \toprule
  \textbf{Var} & \textbf{Value} \\
  \midrule 
  $L_\text{min}$ & 0.1 \si{\metre}\\
  \bottomrule
\end{longtable*}
\end{table}

\begin{table}[!h]
\caption{Output Variables} \label{TblOutputVar}
\renewcommand{\arraystretch}{1.2}
\noindent \begin{longtable*}{l l} 
  \toprule
  \textbf{Var} & \textbf{Physical Constraints} \\
  \midrule 
  $T_W$ & $T_\text{init} \leq T_W \leq T_C$ (by~\aref{A_charge})
  \\
  \bottomrule
\end{longtable*}
\end{table}

\subsubsection{Properties of a Correct Solution} \label{sec_CorrectSolution}

\noindent
A correct solution must satisfy the system of non-linear equations described by \spc{FILL IN}.  

\section{Requirements}

This section provides the functional requirements, the business tasks that the
software is expected to complete, and the nonfunctional requirements, the
qualities that the software is expected to exhibit.

\subsection{Functional Requirements}

\noindent \begin{itemize}

\item[R\refstepcounter{reqnum}\thereqnum \label{R_Inputs}:] \wss{Requirements
    for the inputs that are supplied by the user.  This information has to be
    explicit.}

\item[R\refstepcounter{reqnum}\thereqnum \label{R_OutputInputs}:] \wss{It isn't
    always required, but often echoing the inputs as part of the output is a
    good idea.}

\item[R\refstepcounter{reqnum}\thereqnum \label{R_Calculate}:] \wss{Calculation
    related requirements.}

\item[R\refstepcounter{reqnum}\thereqnum \label{R_VerifyOutput}:]
  \wss{Verification related requirements.}

\item[R\refstepcounter{reqnum}\thereqnum \label{R_Output}:] \wss{Output related
    requirements.}

\end{itemize}

\subsection{Nonfunctional Requirements}

\wss{List your nonfunctional requirements.  You may consider using a fit
  criterion to make them verifiable.}

\section{Likely Changes}    

\noindent \begin{itemize}

\item[LC\refstepcounter{lcnum}\thelcnum\label{LC_meaningfulLabel}:] \wss{Give
    the likely changes, with a reference to the related assumption (aref), as appropriate.}

\end{itemize}

\begin{comment}

\section{Traceability Matrices and Graphs}

The purpose of the traceability matrices is to provide easy references on what
has to be additionally modified if a certain component is changed.  Every time a
component is changed, the items in the column of that component that are marked
with an ``X'' may have to be modified as well.  Table~\ref{Table:trace} shows the
dependencies of theoretical models, general definitions, data definitions, and
instance models with each other. Table~\ref{Table:R_trace} shows the
dependencies of instance models, requirements, and data constraints on each
other. Table~\ref{Table:A_trace} shows the dependencies of theoretical models,
general definitions, data definitions, instance models, and likely changes on
the assumptions.

\wss{You will have to modify these tables for your problem.}

\afterpage{
\begin{landscape}
\begin{table}[h!]
\centering
\begin{tabular}{|c|c|c|c|c|c|c|c|c|c|c|c|c|c|c|c|c|c|c|c|}
\hline
	& \aref{A_OnlyThermalEnergy}& \aref{A_hcoeff}& \aref{A_mixed}& \aref{A_tpcm}& \aref{A_const_density}& \aref{A_const_C}& \aref{A_Newt_coil}& \aref{A_tcoil}& \aref{A_tlcoil}& \aref{A_Newt_pcm}& \aref{A_charge}& \aref{A_InitTemp}& \aref{A_OpRangePCM}& \aref{A_OpRange}& \aref{A_htank}& \aref{A_int_heat}& \aref{A_vpcm}& \aref{A_PCM_state}& \aref{A_Pressure} \\
\hline
\tref{T_COE}        & X& & & & & & & & & & & & & & & & & & \\ \hline
\tref{T_SHE}        & & & & & & & & & & & & & & & & & & & \\ \hline
\tref{T_LHE}        & & & & & & & & & & & & & & & & & & & \\ \hline
\dref{NL}           & & X& & & & & & & & & & & & & & & & & \\ \hline
\dref{ROCT}         & & & X& X& X& X& & & & & & & & & & & & & \\ \hline
\ddref{FluxCoil}    & & & & & & & X& X& X& & & & & & & & & & \\ \hline
\ddref{FluxPCM}     & & & X& X& & & & & & X& & & & & & & & & \\ \hline
\ddref{D_HOF}       & & & & & & & & & & & & & & & & & & & \\ \hline
\ddref{D_MF}        & & & & & & & & & & & & & & & & & & & \\ \hline
\iref{ewat}         & & & & & & & & & & & X& X& & X& X& X& & & X \\ \hline
\iref{epcm}         & & & & & & & & & & & & X& X& & & X& X& X& \\ \hline
\iref{I_HWAT}       & & & & & & & & & & & & & & X& & & & & X \\ \hline
\iref{I_HPCM}       & & & & & & & & & & & & & X& & & & & X & \\ \hline
\lcref{LC_tpcm}     & & & & X& & & & & & & & & & & & & & & \\ \hline
\lcref{LC_tcoil}    & & & & & & & & X& & & & & & & & & & & \\ \hline
\lcref{LC_tlcoil}   & & & & & & & & & X& & & & & & & & & & \\ \hline
\lcref{LC_charge}   & & & & & & & & & & & X& & & & & & & & \\ \hline
\lcref{LC_InitTemp} & & & & & & & & & & & & X& & & & & & & \\ \hline
\lcref{LC_htank}    & & & & & & & & & & & & & & & X& & & & \\
\hline
\end{tabular}
\caption{Traceability Matrix Showing the Connections Between Assumptions and Other Items}
\label{Table:A_trace}
\end{table}
\end{landscape}
}

\begin{table}[h!]
\centering
\begin{tabular}{|c|c|c|c|c|c|c|c|c|c|c|c|c|c|c|c|c|c|c|c|c|c|c|c|}
\hline        
	& \tref{T_COE}& \tref{T_SHE}& \tref{T_LHE}& \dref{NL}& \dref{ROCT} & \ddref{FluxCoil}& \ddref{FluxPCM} & \ddref{D_HOF}& \ddref{D_MF}& \iref{ewat}& \iref{epcm}& \iref{I_HWAT}& \iref{I_HPCM} \\
\hline
\tref{T_COE}     & & & & & & & & & & & & & \\ \hline
\tref{T_SHE}     & & & X& & & & & & & & & & \\ \hline
\tref{T_LHE}     & & & & & & & & & & & & & \\ \hline
\dref{NL}        & & & & & & & & & & & & & \\ \hline
\dref{ROCT}      & X& & & & & & & & & & & & \\ \hline
\ddref{FluxCoil} & & & & X& & & & & & & & & \\ \hline
\ddref{FluxPCM}  & & & & X& & & & & & & & & \\ \hline
\ddref{D_HOF}    & & & & & & & & & & & & & \\ \hline
\ddref{D_MF}     & & & & & & & & X& & & & & \\ \hline
\iref{ewat}      & & & & & X& X& X& & & & X& & \\ \hline
\iref{epcm}      & & & & & X& & X& & X& X& & & X \\ \hline
\iref{I_HWAT}    & & X& & & & & & & & & & & \\ \hline
\iref{I_HPCM}    & & X& X& & & & X& X& X& & X& & \\
\hline
\end{tabular}
\caption{Traceability Matrix Showing the Connections Between Items of Different Sections}
\label{Table:trace}
\end{table}

\begin{table}[h!]
\centering
\begin{tabular}{|c|c|c|c|c|c|c|c|}
\hline
	& \iref{ewat}& \iref{epcm}& \iref{I_HWAT}& \iref{I_HPCM}& \ref{sec_DataConstraints}& \rref{R_RawInputs}& \rref{R_MassInputs} \\
\hline
\iref{ewat}            & & X& & & & X& X \\ \hline
\iref{epcm}            & X& & & X& & X& X \\ \hline
\iref{I_HWAT}          & & & & & & X& X \\ \hline
\iref{I_HPCM}          & & X& & & & X& X \\ \hline
\rref{R_RawInputs}     & & & & & & & \\ \hline
\rref{R_MassInputs}    & & & & & & X& \\ \hline
\rref{R_CheckInputs}   & & & & & X& & \\ \hline
\rref{R_OutputInputs}  & X& X& & & & X& X \\ \hline
\rref{R_TempWater}     & X& & & & & & \\ \hline 
\rref{R_TempPCM}       & & X& & & & & \\ \hline
\rref{R_EnergyWater}   & & & X& & & & \\ \hline
\rref{R_EnergyPCM}     & & & & X& & & \\ \hline
\rref{R_VerifyOutput}  & & & X& X& & & \\ \hline
\rref{R_timeMeltBegin} & & X& & & & & \\ \hline
\rref{R_timeMeltEnd}   & & X& & & & & \\ 
\hline
\end{tabular}
\caption{Traceability Matrix Showing the Connections Between Requirements and Instance Models}
\label{Table:R_trace}
\end{table}

\end{comment}
\newpage

\bibliographystyle {plainnat}
\bibliography {../../ReferenceMaterial/References}

\newpage

\section{Appendix}

\wss{Your report may require an appendix.  For instance, this is a good point to
show the values of the symbolic parameters introduced in the report.}

\subsection{Symbolic Parameters}

\wss{The definition of the requirements will likely call for SYMBOLIC\_CONSTANTS.
Their values are defined in this section for easy maintenance.}


\end{document}