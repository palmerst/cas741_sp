\documentclass[12pt]{article}

\usepackage{amsmath, mathtools}
\usepackage{amsfonts}
\usepackage{amssymb}
\usepackage{graphicx}
\usepackage{colortbl}
\usepackage{xr}
\usepackage{hyperref}
\usepackage{longtable}
\usepackage{xfrac}
\usepackage{tabularx}
\usepackage{float}
\usepackage{siunitx}
\usepackage{booktabs}
\usepackage{caption}
\usepackage{pdflscape}
\usepackage{afterpage}
\usepackage{tabu}
\usepackage{verbatim}
\usepackage[round]{natbib}
\usepackage{url}

\captionsetup{belowskip=12pt,aboveskip=4pt}

\makeatletter
\newcommand*\bigcdot{\mathpalette\bigcdot@{.7}}
\newcommand*\bigcdot@[2]
  {\mathbin{\vcenter{\hbox{\scalebox{#2}{$\m@th#1\bullet$}}}}}
\makeatother

%\usepackage{refcheck}

\hypersetup{
    bookmarks=true,         % show bookmarks bar?
      colorlinks=true,      % false: boxed links; true: colored links
    linkcolor=red,          % color of internal links 
                            %  (change box color with linkbordercolor)
    citecolor=blue,        % color of links to bibliography
    filecolor=magenta,      % color of file links
    urlcolor=cyan           % color of external links
}

%% Comments

\usepackage{color}

\newif\ifcomments\commentstrue

\ifcomments
\newcommand{\authornote}[3]{\textcolor{#1}{[#3 ---#2]}}
\newcommand{\todo}[1]{\textcolor{red}{[TODO: #1]}}
\else
\newcommand{\authornote}[3]{}
\newcommand{\todo}[1]{}
\fi

\newcommand{\wss}[1]{\authornote{blue}{SS}{#1}}
\renewcommand{\sp}[1]{\authornote{magenta}{SP}{#1}}



\newcommand{\sskip}{\vskip 1mm}

% For easy change of table widths
\newcommand{\colZwidth}{1.0\textwidth}
\newcommand{\colAwidth}{0.13\textwidth}
\newcommand{\colBwidth}{0.82\textwidth}
\newcommand{\colCwidth}{0.1\textwidth}
\newcommand{\colDwidth}{0.05\textwidth}
\newcommand{\colEwidth}{0.8\textwidth}
\newcommand{\colFwidth}{0.17\textwidth}
\newcommand{\colGwidth}{0.5\textwidth}
\newcommand{\colHwidth}{0.28\textwidth}

% Used so that cross-references have a meaningful prefix
\newcounter{defnum} %Definition Number
\newcommand{\dthedefnum}{GD\thedefnum}
\newcommand{\dref}[1]{GD\ref{#1}}
\newcounter{datadefnum} %Datadefinition Number
\newcommand{\ddthedatadefnum}{DD\thedatadefnum}
\newcommand{\ddref}[1]{DD\ref{#1}}
\newcounter{theorynum} %Theory Number
\newcommand{\tthetheorynum}{T\thetheorynum}
\newcommand{\tref}[1]{T\ref{#1}}
\newcounter{tablenum} %Table Number
\newcommand{\tbthetablenum}{T\thetablenum}
\newcommand{\tbref}[1]{TB\ref{#1}}
\newcounter{assumpnum} %Assumption Number
\newcommand{\atheassumpnum}{P\theassumpnum}
\newcommand{\aref}[1]{A\ref{#1}}
\newcounter{goalnum} %Goal Number
\newcommand{\gthegoalnum}{P\thegoalnum}
\newcommand{\gsref}[1]{GS\ref{#1}}
\newcounter{instnum} %Instance Number
\newcommand{\itheinstnum}{IM\theinstnum}
\newcommand{\iref}[1]{IM\ref{#1}}
\newcounter{reqnum} %Requirement Number
\newcommand{\rthereqnum}{P\thereqnum}
\newcommand{\rref}[1]{R\ref{#1}}
\newcounter{nfreqnum} %NF Requirement Number
\newcommand{\rthenfreqnum}{P\thenfreqnum}
\newcommand{\nfref}[1]{NF\ref{#1}}
\newcounter{lcnum} %Likely change number
\newcommand{\lthelcnum}{LC\thelcnum}
\newcommand{\lcref}[1]{LC\ref{#1}}
\newcommand{\sref}[1]{\S~\ref{#1}}

\newcommand{\progname}{SpecGen} % PUT YOUR PROGRAM NAME HERE

\usepackage{fullpage}

\begin{document}
\pagenumbering{gobble}

\title{CAS 741: SRS\\[10pt]\Large Aqueous Speciation Diagram Generator}
\author{Steven Palmer\\\texttt{palmes4}}
\date{\today}
	
\maketitle

~\newpage

\pagenumbering{roman}

\setcounter{secnumdepth}{0}

\section{Revision History}

\begin{table}[hp]
\caption{Revision History} \label{TblRevisionHistory}
\begin{tabularx}{\textwidth}{llX}
\toprule
\textbf{Date} & \textbf{Developer(s)} & \textbf{Change}\\
\midrule
10.5.2017 & S. Palmer & First revision of document\\
12.18.2017 & S. Palmer & Revision 1\\
\bottomrule
\end{tabularx}
\end{table}

~\newpage

\section{Reference Material}

This section records information for easy reference.

\subsection{Notation}
This section describes the notation conventions used in this document. 

\subsubsection{Chemical Notation}
The following notation is used to describe chemical reactions:\\

${\alpha}_1 {A}_{1} + \cdots + {\alpha}_n {A}_{n} 
    \rightleftharpoons {\beta}_1 {B}_{1} + \cdots + {\beta}_m {B}_{m}$\\

where $A_i$ are reactant species, $\alpha_i$ are the stoichiometric numbers
of the reactant species, $n$ is the number of reactant species, $B_i$ are product species,
$\beta_i$ are the stoichiometric numbers of the product species, $m$ is the number of
product species.\\

\noindent
Chemical species are denoted as follows:\\

${{E}^1_{e1}}\cdots{{E}^n_{en}}^{z}_{(s)}$\\

where $E^i$ are chemical elements, $e_i$ are the number of each element $E^i$ in
the species, $n$ is the number of distinct elements in the species, $z$ is the
charge, and $s$ is the state.  The state of a species ranges over the values
aqueous (aq), solid (s), liquid (l), or gas (g).

\subsubsection{Mathematical Notation}
In addition to standard mathematical notation conventations, the Gries and Schneider 
notations~\cite{GS1993} for 
set building and the evaluation of
operators applied over sets are use extensively in this document.  
Specifically, the expression $\{x: X~|~R~\bigcdot~P\}$ means the set of all values $x$ 
that are members of $X$ for which the range $R$ is true and the predicate $P$ is satisfied.  
A set preceded by $\Sigma$ means the sum of all values in that set.

\subsubsection{Spaces}
All symbols that occur in this document have an associated space (a set which their values 
belong to).  Standard mathematical spaces are used (e.g. $\mathbb{N}$, $\mathbb{Z}$, 
$\mathbb{R}$, etc.) as well as some additional spaces defined in the following table. 
~\newline

\renewcommand{\arraystretch}{1.2}
%\begin{table}[ht]
  \noindent \begin{tabular}{l l} 
    \toprule		
    \textbf{symbol} & \textbf{description}\\
    \midrule 
    Sp. & chemical species \\
    Rx. & chemical reaction \\
    Elt. & chemical element \\
    \bottomrule
  \end{tabular}
  %	\caption{Provide a caption}
%\end{table}


\subsection{Table of Units}

Throughout this document SI (Syst\`{e}me International d'Unit\'{e}s) is employed 
as the unit system.  In addition to the basic units, several derived units are
used as described below.  For each unit, the symbol is given followed by a
description of the unit and the SI name.
~\newline

\renewcommand{\arraystretch}{1.2}
%\begin{table}[ht]
  \noindent \begin{tabular}{l l l} 
    \toprule		
    \textbf{symbol} & \textbf{unit} & \textbf{SI}\\
    \midrule 
    \si{\metre} & length & metre\\
    \si{\kilogram} & mass & kilogram\\
    \si{\second} & time & second\\
    \si{\celsius} & temperature & centigrade\\
    \si{\mole} & amount of substance & mole\\
    \si{\litre} & volume & litre 
      ($\si{\litre} = 10^{-3}\cdot\si{\cubic\metre}$)\\
    \si{\pascal} & pressure & pascal 
      ($\si{\pascal} = \si{\kilogram\per\metre\per\square\second}$)\\
    \si{\bar} & pressure & bar ($\si{\bar} = 10^{5}\cdot\si{\pascal}$)\\
    \bottomrule
  \end{tabular}
  %	\caption{Provide a caption}
%\end{table}

\subsection{Table of Symbols}

The table that follows summarizes the symbols used in this document along with
their units.  The choice of symbols was made to be consistent with chemistry 
literature.  The symbols are listed in alphabetical order.
~\newline
\renewcommand{\arraystretch}{1.2}
%\noindent \begin{tabularx}{1.0\textwidth}{l l X}
\noindent \begin{longtable*}{l l l p{8cm}} \toprule
\textbf{symbol} & \textbf{space} & \textbf{unit} & \textbf{description}\\
\midrule 
$\alpha$ & $\mathbb{N}$ & -- & stoichiometric coefficient of reactant species
\\
$\beta$ & $\mathbb{N}$ & -- & stoichiometric coefficient of product species
\\$A$ & Sp. & -- & reactant species in a chemical equation
\\
$B$ & Sp. & -- & product species in a chemical equation
\\ 
$i$ & $\mathbb{N}$ & -- & index counter
\\ 
$j$ & $\mathbb{N}$ & -- & index counter
\\
$K$ & $\mathbb{R}$ & -- & equilibrium constant
\\
$K_{H_2O}$ & $\mathbb{R} \rightarrow \mathbb{R}$ & -- & equilibrium equation for the dissociation of water
\\
$\overline K$ & Set of $\mathbb{R}^n \rightarrow \mathbb{R}$ & -- & set of equilibria equations 
\\
$k_+$ & $\mathbb{R}$ & \si{\mole\per\second} & forward rate constant
\\
$k_-$ & $\mathbb{R}$ & \si{\mole\per\second} & backward rate constant
\\
$\overline M$ & Set of $\mathbb{R}^n \rightarrow \mathbb{R}$ & -- &  set of mass balance equations
\\ 
$m$ & $\mathbb{N}$ & -- & index counter
\\ 
$n$ & $\mathbb{N}$ & -- & index counter
\\
$P$ & $\mathbb{R}$ & \si{\bar} & pressure
\\
$R$ & Rx. & -- & an arbitrary reversible reaction
\\ 
$\overline {RK}_{in}$ & Set of (Rx., $\mathbb{R}$) & -- & the input set of all pairs of reactions and 
  associated equilibrium constants 
\\ 
$r_+$ & $\mathbb{R}$ & \si{\mole\per\second} & forward reaction rate
\\
$r_-$ & $\mathbb{R}$ & \si{\mole\per\second} & backward reaction rate
\\
$\overline S$ & Set of Sp.  & -- & a set of aqueous solute species in a chemical system
\\ 
$\overline S_{in}$ & Set of Sp. & -- & the input set of all aqueous solute species in 
  the chemical system
\\ 
$T$ & $\mathbb{R}$ & $^{\circ}$C & temperature
\\
$\overline T_{in}$ & Set of $\mathbb{R}$ & -- & the input set of total concentrations of chemical 
  elements
\\ 
$X$ & Sp. & -- & arbitrary chemical species
\\
$x$ & Elt. & -- & arbitrary chemical element
\\
$x_{tot}$ & $\mathbb{R}$ & \si[per-mode=symbol] {\mole\per\litre} & total concentration of an 
  arbitrary chemical element $x$
\\
$[X]$ & $\mathbb{R}$ & \si[per-mode=symbol] {\mole\per\litre} & concentration of the arbitrary 
  chemical species $X$
\\
$\{X\}$ & $\mathbb{R}$ & -- & activity of the arbitrary chemical species $X$
\\
$z_{X}$ & $\mathbb{Z}$ & -- & charge number of an arbitrary aqueous solute $X$
\\
$\#x\langle X\rangle$ & $\mathbb{N}$ & -- & the number of occurrences of $x$ in $X$
\\
\bottomrule
\end{longtable*}


\subsection{Abbreviations and Acronyms}

\renewcommand{\arraystretch}{1.2}
\begin{tabular}{l l} 
  \toprule		
  \textbf{symbol} & \textbf{description}\\
  \midrule 
  A & Assumption\\
  DD & Data Definition\\
  GD & General Definition\\
  GS & Goal Statement\\
  IM & Instance Model\\
  LC & Likely Change\\
  NF & Non-Functional Requirement\\
  PS & Physical System Description\\
  R & Requirement\\
  SATP & Standard Ambient Temperature and Pressure\\
  SRS & Software Requirements Specification\\
  \progname{} & The Aqueous Speciation Diagram Generator program\\
  T & Theoretical Model\\
  \bottomrule
\end{tabular}\\

\newpage

\tableofcontents

~\newpage

\pagenumbering{arabic}

\setcounter{secnumdepth}{3}

\section{Introduction}

\subsection{Purpose of Document}
The purpose of this document is to describe the requirements for the Aqueous 
Speciation Diagram Generator program (herein referred to as \progname), a 
software product that will produce a speciation diagram given a set of chemical 
reactions and element totals that define a chemical system as inputs.  \progname{}
is reimplementation of a MATLAB implementation by~\cite{smith2007} 
(Wilfrid Laurier University) using a rational design process.  The goals 
and theoretical models used in the SpecGen code are provided, with an emphasis 
on explicitly identifying assumptions and unambiguous definitions. This document 
is intended to be used as a reference to provide all information necessary to 
understand and verify the transformation of inputs to outputs. The SRS is 
abstract: the contents describe the problem being solved, but not how to solve 
it.

This document will be used as a starting point for subsequent development
phases, including writing the design specification and the software verification
and validation plan. The design document will show how the requirements are to
be realized, including decisions on the numerical algorithms and programming
environment. The verification and validation plan will show the steps that will
be used to increase confidence in the software documentation and the
implementation. Although the SRS fits in a series of documents that follow the
so-called waterfall model, the actual development process is not constrained in
any way. Even when the waterfall model is not followed, as
\cite{ParnasAndClements1986} point out, the most logical way to present the
documentation is still to ``fake'' \wss{Use ``fake'' to get correct quotation
  marks} a rational design process.

\wss{The text is better for version control, and for reading in other editors,
  if you use a hard-wrap at 80 characters}

\subsection{Scope of Requirements} 
\spc{Note: Think about what should actually go here for revision 1.  Major
  assumptions?} \wss{The first sentence below reads like a restating of what the
  program does.  I think we say this enough already.  I like the idea of the
  scope listing the ``major'' assumptions.  I think you are right about pH being
an problem defining assumption.  What about the fact that the system is in an
aqueous environment?  The SATP assumption also seems like a scope level
assumption to me.  Maybe ``scope assumptions'' are those that need to be made to
build the theory models?  Assumptions that are subsequently used to build the
instance model would then be outside of the scope of scope.  We can test this
idea by applying it to our examples and seeing which assumptions would be
classified as scope assumptions.}
The scope of \progname{} is limited to the generation of speciation diagrams
for aqueous chemical systems.  Analysis of gas and solid phases will not be
considered.  The chemical systems under investigation are always assumed to be
at Standard Ambient Temperature and Pressure (SATP).  Diagrams generated by 
\progname{} will plot aqueous species concentrations over the pH range 0 to
14.  \wss{Did you define pH anywhere?  I know it is basic knowledge (acidic
  knowledge? :-) ), but I think it would add to the document.}

\subsection{Characteristics of Intended Reader} 
Reviewers of this documentation should have an understanding of first year 
undergraduate physical chemistry.


\subsection{Organization of Document}
The organization of this document follows the template for an SRS for scientific 
computing software proposed by \cite{SmithAndLai2005} and \cite{Koothoor2013}, 
with some aspects taken from \cite{RobertsonAndRobertson1999Vol}. The 
presentation follows the standard pattern of presenting goals, theories, 
definitions, and assumptions. For readers that would like a more bottom up 
approach, they can start reading the data definitions in Sec:DataDefi and trace 
back to find any additional information they require.

The goal statements are refined to the theoretical models, and the theoretical 
models to the instance models. The data definitions are used to support the 
definitions of the different models.

\section{General System Description}
This section identifies the interfaces between the system and its environment,
describes the user characteristics and lists the system constraints.

\subsection{System Context}

\begin{itemize}
\item User Responsibilities:
\begin{itemize}
\item Ensure that the input data is correct.
\item Ensure that the input data is sufficient to specify a chemical system 
that has a unique speciation solution.
\end{itemize}
\item \progname{} Responsibilities:
\begin{itemize}
\item Detect data type mismatch, such as a string of characters instead of a
  floating point number.
\item Determine if the inputs satisfy the required physical and software 
  constraints.
\item Solve the system of equations arising from the input data to generate 
  the output data.
\item Generate a plot of the output data.
\end{itemize}
\end{itemize}

\subsection{User Characteristics} \label{SecUserCharacteristics}
The end user of \progname{} should have an understanding of first year 
undergraduate physical chemistry.

\subsection{System Constraints}
There are no system constraints.

\section{Specific System Description}

This section first presents the problem description, which gives a high-level
view of the problem to be solved.  This is followed by the solution 
characteristics specification, which presents the assumptions, theories, 
definitions and finally the instance models.

\subsection{Problem Description} \label{Sec_pd}
Chemical speciation refers to the stable (equilibrium) distribution of chemical 
species in a given chemical system. Speciation diagrams, which plot species 
concentrations against an independently varied parameter of the system, are 
useful tools for displaying speciation data in a concise and easy to use format. 
The production of speciation diagrams requires solving a set of non-linear 
equations which arise from the chemical reactions of the species present in a 
chemical system.  As the number of reactions taking place in a system increases, 
producing speciation diagrams can quickly become a tedious undertaking when done 
manually.

\progname{} will produce a speciation diagram given a set of chemical reactions, 
equilibrium constants, and element totals that define a chemical system. 
\progname{} will be specific to speciation of ions in aqueous systems under 
varying pH, which is of particular importance in the fields of aqueous process 
engineering and hydrometallurgy.  The diagram generated by \progname{} will 
plot speciation of all aqueous species (excluding H$^+$ and OH$^-$) across the 
pH range 0 to 14.

\subsubsection{Terminology and Definitions}

This subsection provides a list of terms that are used in the subsequent
sections and their meaning, with the purpose of reducing ambiguity and making it
easier to correctly understand the requirements:

\begin{description}
\item[Activity:] the ``effective concentration'' of a species in a chemical 
  system
\item[Equilibrium:] the state of a chemical system in which species 
  concentrations are stable
\item[pH:] Potential of Hydrogen:  pH $= -\log{\{H^+\}}$
\item[SATP:] Standard Ambient Temperature and Pressure
  $(T = 25^{\circ}\textrm{C}, ~P = 1~\textrm{bar})$ \wss{When you define an
    acronym, in the expanded form you should capitalize the letters that are
    being used.} \wss{$P$ is not in your list of symbols.}
\end{description}

\subsubsection{Physical System Description}

The physical system of \progname{} includes the following elements:

\begin{itemize}
\item[PS1:] An unspecified volume of liquid water containing dissolved chemical 
  species under equilibrium conditions.
\end{itemize}


\subsubsection{Goal Statements}

\noindent Given a set of chemical species present in an aqueous chemical system, 
a set of reversible reactions and associated equilibrium constants that the 
species are subject to, and the total concentrations of all elements present in 
the species set (excluding O and H), the goal statements are:

\begin{itemize}

\item[GS\refstepcounter{goalnum}\thegoalnum \label{G_plot}:] Generate a plot of 
  the equilibrium concentrations of all chemical species present in the system 
  versus pH over the pH range 0 to 14.
\end{itemize}

\wss{The more I think about it, the more I think you really only have one goal -
  to generate the plot.  The calculations are implied by the need to generate
  the plot.  Maybe we can start thinking about goals as the outputs of the
  software?  In general goal statements can characterize at a high level
  (abstract level) the inputs and the outputs.}

\subsection{Solution Characteristics Specification}

The instance models that govern \progname{} are presented in 
\sref{sec_instance}. The information to understand the meaning of the instance 
models and their derivation is also presented, so that the instance models can 
be verified.

\subsubsection{Assumptions}

This section simplifies the original problem and helps in developing the
theoretical model by filling in the missing information for the physical
system. The numbers given in the square brackets refer to the theoretical model
[T], general definition [GD], data definition [DD], instance model [IM], or
likely change [LC], in which the respective assumption is used.

\begin{itemize}
\item[A\refstepcounter{assumpnum}\theassumpnum \label{A_pH}:]
  All generated speciation diagrams will cover the pH range 0 to 14. 
  [\iref{IM_solve}]
\item[A\refstepcounter{assumpnum}\theassumpnum \label{A_SATP}:]
  All generated speciation diagrams are at SATP conditions.  It follows that 
  all user supplied equilibrium constants are assumed to be values measured 
  under SATP conditions. [\dref{GD_water}, \lcref{LC_SATP}]
\item[A\refstepcounter{assumpnum}\theassumpnum \label{A_only_aq}:]
  The chemical system is assumed be a single-solvent solution, where the solvent 
  is liquid water at SATP.  The chemical species in all of the user supplied 
  equations are assumed to be aqueous solutes or water. [\iref{IM_solve}, 
  \lcref{LC_state}]
\item[A\refstepcounter{assumpnum}\theassumpnum \label{A_act_wat}:]
  The activity of liquid water is assumed to be 1.0. [\iref{IM_solve}, 
  \lcref{LC_act}]
\item[A\refstepcounter{assumpnum}\theassumpnum \label{A_act_aq}:]
  The activities of all aqueous solute species are assumed to be equal to 
  their concentration. [\iref{IM_solve}, \lcref{LC_act}]
\item[A\refstepcounter{assumpnum}\theassumpnum \label{A_rev}:]
  All chemical equations supplied by the user are assumed to be reversible 
  (finite equilibrium constant). [\ddref{DD_setK}]
\end{itemize}

\subsubsection{Theoretical Models}\label{sec_theoretical}

This section focuses on the general equations and laws that \progname{} is based
on.

~\newline
\noindent
\begin{minipage}{\textwidth}
\renewcommand*{\arraystretch}{1.5}
\tabulinesep=1.5mm
\begin{tabu}{| p{\colAwidth} | p{\colBwidth}|}
  \hline
  \rowcolor[gray]{0.9}
  Number& T\refstepcounter{theorynum}\thetheorynum \label{T_MB}\\
  \hline
  Label&\bf Conservation of mass (concentration balance)\\
  \hline
  Equation&  $x_{tot} = \sum\{X : \overline S ~|~ x \in X ~\bigcdot~ 
    \#x\langle X\rangle \cdot [X]\}$\\
  \hline
  Description & \wss{I think you need to explain your mathematical notation.  I
    know you had the idea of a pointer to a description for Z, but you could
    likely summarize the notation in the document itself, without mentioning Z.
    The notation you are using is like the Gries and Schneider notation.
    Something like: The above equations use the Gries and Schneider
notation~\cite[p. 143]{GriesAndSchneider1993} for set building and evaluation of
an operator applied over a set of values.  Specifically, the expression
$(*x: X | R : P)$ means application of the operator $*$ to the values $P$ for
all $x$ of type $X$ for which range $R$ is true.  In the above equations, the
$*$ operators ...}
                The total dissolved concentration of an element in an aqueous 
                  solution is equal to the sum of the concentrations of all 
                  species which contain that element.  This is captured in the 
                  above equation, where:\\
              & $x$ is a chemical element\\
              & $x_{tot}$ is the total dissolved concentration of $x$ in a 
                  solution\\
              & $X$ is an aqueous species in solution\\
              & $\overline S$ is a set of all aqueous species in a solution\\
              & $\#x\langle X\rangle$ is the number of occurrences of $x$ in 
                  $X$\\
              & $[X]$ is the concentration of $X$ in solution\\
  \hline
  Source &--\\
  \hline
  Ref.\ By & \ddref{DD_setM}\\
  \hline
\end{tabu}
\end{minipage}\\

~\newline
\noindent
\begin{minipage}{\textwidth}
\renewcommand*{\arraystretch}{1.5}
\tabulinesep=1.5mm
\begin{tabu}{| p{\colAwidth} | p{\colBwidth}|}
  \hline
  \rowcolor[gray]{0.9}
  Number& T\refstepcounter{theorynum}\thetheorynum \label{T_LMAF}\\
  \hline
  Label &\bf Law of mass action (forward) \\
  \hline
  Equation&$ r_+ = k_+ \displaystyle\prod_{i=1}^{n} \{A_i\}^{\alpha_i} $  \\
  \hline
  Description &
                The law of mass action describes the forward and backward 
                  reaction rates for a chemical reaction.  For a given
                  chemical reaction, the equation above expresses the law of mass 
                  action in the forward direction, where:\\
              & $r_+$ is the forward reaction rate (\si{\mole\per\second}).\\
              & $k_+$ is the forward rate constant (\si{\mole\per\second}).\\
              & $n$ is the number of reactants.\\
              & $A_i$ is the $i^{th}$ reactant species.\\
              & $\{A_i\}$ is the activity of $A_i$.\\
              & $\alpha_i$ is the stoichiometric coefficient of the 
                  $i^{\textrm{th}}$ reactant.\\
  \hline
  Source &~\cite{wiki:eq}\\
  \hline
  Ref.\ By & \tref{T_EQ}\\
  \hline
\end{tabu}
\end{minipage}\\
~\newline

\noindent
\begin{minipage}{\textwidth}
\renewcommand*{\arraystretch}{1.5}
\tabulinesep=1.5mm
\begin{tabu}{| p{\colAwidth} | p{\colBwidth}|}
  \hline
  \rowcolor[gray]{0.9}
  Number& T\refstepcounter{theorynum}\thetheorynum \label{T_LMAB}\\
  \hline
  Label &\bf Law of mass action (backward) \\
  \hline
  Equation&$ r_- = k_- \displaystyle\prod_{j=1}^{m} \{B_j\}^{\beta_j} $ \\
  \hline
  Description &
                The law of mass action describes the forward and backward 
                  reaction rates for a chemical reaction.  For a given
                  chemical reaction, the equation above expresses the law of mass 
                  action in the backward direction, where:\\
              & $r_-$ is the backward reaction rate (\si{\mole\per\second}).\\
              & $k_-$ is the backward rate constant (\si{\mole\per\second}).\\
              & $m$ is the number of products.\\
              & $B_j$ is the $j^{th}$ product species.\\
              & $\{B_j\}$ is the activity of $B_j$.\\
              & $\beta_j$ is the stoichiometric coefficient of the 
                  $j^{\textrm{th}}$ product.\\
  \hline
  Source &~\cite{wiki:eq}\\
  \hline
  Ref.\ By & \tref{T_EQ}\\
  \hline
\end{tabu}
\end{minipage}\\
~\newline

\noindent
\begin{minipage}{\textwidth}
\renewcommand*{\arraystretch}{1.5}
\tabulinesep=1.5mm
\begin{tabu}{| p{\colAwidth} | p{\colBwidth}|}
  \hline
  \rowcolor[gray]{0.9}
  Number& T\refstepcounter{theorynum}\thetheorynum \label{T_EQ}\\
  \hline
  Label &\bf Chemical equilibrium \\
  \hline
  Equation&$ k_+ \displaystyle\prod_{i=1}^{n} \{A_i\}^{\alpha_i} = k_- 
    \displaystyle\prod_{j=1}^{m} \{B_j\}^{\beta_j} $ \\
  \hline
  Description &
                When a chemical reaction is in equilibrium, the forward and 
                  reverse reaction rates are the same.  Using \tref{T_LMAF} and 
                  \tref{T_LMAB}, this idea is expressed in the equation above, 
                  where:\\
              & $k_+$ is the forward rate constant (\si{\mole\per\second}).\\
              & $n$ is the number of reactants.\\
              & $A_i$ is the $i^{th}$ reactant species.\\
              & $\{A_i\}$ is the activity of $A_i$.\\
              & $\alpha_i$ is the stoichiometric coefficient of the 
                  $i^{\textrm{th}}$ reactant.\\
              & $k_-$ is the backward rate constant (\si{\mole\per\second}).\\
              & $m$ is the number of products.\\
              & $B_j$ is the $j^{th}$ product species.\\
              & $\{B_j\}$ is the activity of $B_j$.\\
              & $\beta_j$ is the stoichiometric coefficient of the 
                  $j^{\textrm{th}}$ product.\\
  \hline
  Source &~\cite{wiki:eq}\\
  \hline
  Ref.\ By & \tref{T_K}\\
  \hline
\end{tabu}
\end{minipage}\\

~\newline

\noindent
\begin{minipage}{\textwidth}
\renewcommand*{\arraystretch}{1.5}
\tabulinesep=1.5mm
\begin{tabu}{| p{\colAwidth} | p{\colBwidth}|}
\hline
\rowcolor[gray]{0.9}
Number& T\refstepcounter{theorynum}\thetheorynum \label{T_K}\\
\hline
Label& \bf Equilibrium equation\\
\hline
% Units& $Mt^{-3}$\\
% \hline
  Equation&$K = \dfrac{r_+}{r_-} = \dfrac{\prod_{j=1}^{m} \{B_j\}^{\beta_j}}
    {\prod_{i=1}^{n} \{A_i\}^{\alpha_i}}$\\
  \hline
  Description & 
                The ratio of the forward and backward rate constants for a 
                  reaction forms a new constant called the equilibrium constant.  
                  Using \tref{T_EQ}, we can relate the equilibrium constant to 
                  the activities of the products and reactants.  This is shown 
                  in the equation above, where:\\
              & $K$ is the equilibrium constant.\\
              & $r_+$ is the forward reaction rate (\si{\mole\per\second}).\\
              & $k_+$ is the forward rate constant (\si{\mole\per\second}).\\
              & $n$ is the number of reactants.\\
              & $A_i$ is the $i^{th}$ reactant species.\\
              & $\{A_i\}$ is the activity of $A_i$.\\
              & $\alpha_i$ is the stoichiometric coefficient of the 
                $i^{\textrm{th}}$ reactant.\\
              & $r_-$ is the backward reaction rate (\si{\mole\per\second}).\\
              & $k_-$ is the backward rate constant (\si{\mole\per\second}).\\
              & $m$ is the number of products.\\
              & $B_j$ is the $j^{th}$ product species.\\
              & $\{B_j\}$ is the activity of $B_j$.\\
              & $\beta_j$ is the stoichiometric coefficient of the 
                $j^{\textrm{th}}$ product.\\
  \hline
  Sources&~\cite{wiki:eq}  \\
  \hline
  Ref.\ By & \dref{GD_water}, \ddref{DD_setK} \\
  \hline
\end{tabu}
\end{minipage}\\

\subsubsection{General Definitions}\label{sec_gendef}

This section collects the laws and equations that will be used in deriving the
data definitions, which in turn are used to build the instance models.

~\newline

\noindent
\begin{minipage}{\textwidth}
\renewcommand*{\arraystretch}{1.5}
\tabulinesep=1.5mm
\begin{tabu}{| p{\colAwidth} | p{\colBwidth}|}
  \hline
  \rowcolor[gray]{0.9}
  Number& GD\refstepcounter{defnum}\thedefnum \label{GD_water}\\
  \hline
  Label &\bf Water dissociation equilibrium \\
  \hline
  SI Units&--\\
  \hline
  Equation&$ K_{H_2O}:~10^{-14} = \{H^+\} \cdot \{OH^-\} $  \\
  \hline
  Description &
  This is the equilibrium equation for the dissociation of water.  The equation 
  arises from \tref{T_K} using the value of $10^{-14}$ for the equilibrium 
  constant at SATP, where:\\
  &$\{H^+\}$ is the activity of $H^+$\\
  &$\{OH^-\}$ is the activity of $OH^-$\\
  \hline
  Source &~\cite{wiki:water}\\
  \hline
  Ref.\ By & \ddref{DD_setK}\\
  \hline
\end{tabu}
\end{minipage}\\


\subsubsection{Data Definitions}\label{sec_datadef}

This section collects and defines all the data needed to build the instance
models. The dimension of each quantity is also given.


~\newline

\noindent
\begin{minipage}{\textwidth}
\renewcommand*{\arraystretch}{1.5}
\tabulinesep=1.5mm
\begin{tabu}{| p{\colAwidth} | p{\colBwidth}|}
\hline
\rowcolor[gray]{0.9}
Number& DD\refstepcounter{datadefnum}\thedatadefnum \label{DD_setK}\\
\hline
Label& \bf Set of equilibria \wss{I like the idea of adding fields for types to
  your problem.}\\
\hline
Symbol &$\overline K$\\
\hline
% Units& $Mt^{-3}$\\
% \hline
  SI Units & --\\
  \hline
  Equation&$\overline K = \left\{(R,K_R):\overline{RK}_{in} ~\bigcdot~ K_R = 
    \dfrac{\prod_{j=1}^{m} \{B_{j}\}^{\beta_{j}}}{\prod_{i=1}^{n} 
    \{A_{i}\}^{\alpha_{i}}}~\textrm{with reaction}~R\right\} \cup K_{H_2O}$\\
  \hline
  Description & By \aref{A_rev}, all of the input reactions are reversible with 
    finite equilibrium constants.  Thus, we can express the input reactions as a 
    set of equilibria via \tref{T_K}. The equilibrium equation for the 
    dissociation of water (\dref{GD_water}) is added to this set since it is not 
    supplied by the user.\\
  \hline
  Sources&--  \\
  \hline
  Ref.\ By & \iref{IM_solve}\\
  \hline
\end{tabu}
\end{minipage}\\

~\newline

\noindent
\begin{minipage}{\textwidth}
\renewcommand*{\arraystretch}{1.5}
\tabulinesep=1.5mm
\begin{tabu}{| p{\colAwidth} | p{\colBwidth}|}
\hline
\rowcolor[gray]{0.9}
Number& DD\refstepcounter{datadefnum}\thedatadefnum \label{DD_setM}\\
\hline
Label& \bf Set of mass balance equations\\
\hline
Symbol &$\overline M$\\
\hline
% Units& $Mt^{-3}$\\
% \hline
  SI Units & --\\
  \hline
  Equation&$\overline M = \left\{x_{tot}:\overline T_{in} ~\bigcdot~ \sum\{X : 
    \overline S_{in} ~|~ x \in X ~\bigcdot~ \#x\langle X\rangle \cdot [X]\} = 
    x_{tot}\right\}$\\
  \hline
  Description & Instantiation of \tref{T_MB} using the input sets $\overline 
    T_{in}$ and $\overline S_{in}$ yields a set of mass (concentration) balance 
    equations.\\
  \hline
  Sources&--  \\
  \hline
  Ref.\ By & \iref{IM_solve}\\
  \hline
\end{tabu}
\end{minipage}\\


\subsubsection{Instance Models} \label{sec_instance}    

This section transforms the problem defined in \sref{Sec_pd} into 
one which is expressed in mathematical terms. It uses concrete symbols defined 
in \sref{sec_datadef} to replace the abstract symbols in the models 
identified in \sref{sec_theoretical} and \sref{sec_gendef}.

The goal \gsref{G_plot} is accomplished by plotting the results of
 \iref{IM_solve}.

~\newline

%Instance Model 1

\noindent
\begin{minipage}{\textwidth}
\renewcommand*{\arraystretch}{1.5}
\tabulinesep=1.5mm
\begin{tabu}{| p{\colAwidth} | p{\colBwidth}|}
  \hline
  \rowcolor[gray]{0.9}
  Number& IM\refstepcounter{instnum}\theinstnum \label{IM_solve}\\
  \hline
  Label& \bf System of non-linear equations to find equilibrium concentrations\\
  \hline
  Input&$\overline K$, $\overline M$\\
  \hline
  Output& $\{X:\overline S_{in} ~|~ X \neq OH^- \land X \neq H^+ ~\bigcdot~ 
            \{[X]~\textrm{for}~0 \leq pH \leq 14\} \}$,\\
        & when the system of non-linear equations $\overline K \cup \overline M$
         is satisfied\\
  \hline
  Description & 
                The union of $\overline K$ (\ddref{DD_setK}) and $\overline M$ 
                  (\ddref{DD_setM}) makes up the system 
                  of non-linear equations that must be solved.  By 
                  \aref{A_only_aq}, all activities in the system of equations 
                  $\overline K \cup \overline M$ are either that of water 
                  or that of an aqueous solute species.  By \aref{A_act_wat}, 
                  all instances of $\{H_2O\}$ can be replaced with 1.0.  By 
                  \aref{A_act_aq}, all activities of aqueous solute species can 
                  be replaced by their concentrations.  This results in a 
                  non-linear system of equations where the set of unknowns is 
                  the set $\overline S_{in}$.  By \aref{A_pH}, the system is 
                  solved on the pH range 0 to 14.
  \\
  \hline
  Sources& -- \\
  \hline
  Ref.\ By & --\\
  \hline
\end{tabu}
\end{minipage}\\

%~\newline

\subsubsection{Data Constraints} \label{sec_DataConstraints}    

Tables~\ref{TblInputVar} and \ref{TblOutputVar} show the data constraints on the
input and output variables, respectively.  The column for physical constraints 
gives the physical limitations on the range of values that can be taken by the
variable.  The column for software constraints restricts the range of inputs to
reasonable values.  The constraints are conservative, to give the user of the
model the flexibility to experiment with unusual situations.  The column of
typical values is intended to provide a feel for a common scenario.  The
uncertainty column provides an estimate of the confidence with which the
physical quantities can be measured.  This information would be part of the
input if one were performing an uncertainty quantification exercise.

The specification parameters in Table~\ref{TblInputVar} are listed in
Table~\ref{TblSpecParams}.


\begin{table}[!h]
  \centering
  \caption{Input Variables} \label{TblInputVar}
  \renewcommand{\arraystretch}{1.2}
\noindent \begin{tabular}{l l l l c} 
  \toprule
  \textbf{Input} & \textbf{Physical Constraints} & \textbf{Software Constraints}  
                   & \textbf{Typical Value} & \textbf{Uncertainty}\\
  \midrule 
  $x_{tot}$ & -- & $x_{{tot}_{\text{min}}} \leq x_{tot} $ & 0.1 
    \si[per-mode=symbol] {\mole\per\litre} & 1\%\\
  $R$ & -- & -- & -- & --\\
  $K$ & -- & -- & $10^{-4}$ & 2.5\%\\
  \bottomrule
  \multicolumn{5}{l}{\scriptsize * The user will supply a set of each input 
    variable.}\\
  \multicolumn{5}{l}{\scriptsize ** There must be an accompanying equilibrium 
    constant for every equation supplied by the user.}\\
  \multicolumn{5}{l}{\scriptsize *** The set of pairs $(R, K)$ is 
    $\overline {RK}_{in}$.}\\
  \multicolumn{5}{l}{\scriptsize **** The set of $x_{tot}$ is 
    $\overline {T}_{in}$.}
\end{tabular}
\end{table}


\begin{table}[!h]
\centering
\caption{Specification Parameter Values} \label{TblSpecParams}
\renewcommand{\arraystretch}{1.2}
\noindent \begin{tabular}{l l} 
  \toprule
  \textbf{Var} & \textbf{Value} \\
  \midrule 
  $x_{{tot}_{\text{min}}}$ & 0 \si{\mole\per\litre}\\
  \bottomrule
\end{tabular}
\end{table}

\begin{table}[!h]
\centering
\caption{Output Variables} \label{TblOutputVar}
\renewcommand{\arraystretch}{1.2}
\noindent \begin{tabular}{l c} 
  \toprule
  \textbf{Output} & \textbf{Physical Constraints} \\
  \midrule 
  $\{X:\overline S_{in} ~|~ X \neq OH^- \land X \neq H^+ ~\bigcdot~ 
    \{[X]~\textrm{for}~0 \leq pH \leq 14\} \}$ & $[X] \geq 0$
  \\
  \bottomrule
\end{tabular}
\end{table}

\subsubsection{Properties of a Correct Solution} \label{sec_CorrectSolution}

\noindent
A correct solution must satisfy the system of non-linear equations described by 
\iref{IM_solve}. 

\section{Requirements}

This section provides the functional requirements, the business tasks that the
software is expected to complete, and the nonfunctional requirements, the
qualities that the software is expected to exhibit.

The goal \gsref{G_plot} is solved by \rref{R_Output}.

\subsection{Functional Requirements}

\noindent \begin{itemize}

\item[R\refstepcounter{reqnum}\thereqnum \label{R_Inputs}:] \progname{} will 
  take the following inputs:
  \begin{enumerate} \item $\overline{RK}_{in}$: the set of all pairs of reactions and 
                      associated equilibrium constants 
                    \item $\overline{T}_{in}$: the set of total concentrations of chemical 
                      elements
  \end{enumerate}

\item[R\refstepcounter{reqnum}\thereqnum \label{R_DerivedInputs}:] \progname{} 
  will derive $\overline{S}_{in}$ (the set of all aqueous solute species in the chemical 
  system) from the set of reactions in \rref{R_Inputs}.

\item[R\refstepcounter{reqnum}\thereqnum \label{R_ValidateInputs}:] \progname{} 
  will ensure that the inputs do not violate the constraints specified in 
  \sref{sec_DataConstraints}.

\item[R\refstepcounter{reqnum}\thereqnum \label{R_Calculate}:] \progname{} will 
  calculate equilibrium concentrations for all species in $\overline{S}_{in}$ (excluding 
  $H^+$ and $OH^-$) via \iref{IM_solve}.

\item[R\refstepcounter{reqnum}\thereqnum \label{R_Output}:] \progname{} will 
  generate a speciation diagram of all species in $\overline{S}_{in}$ (excluding $H^+$ and 
  $OH^-$) in the pH range 0 to 14 using the results of the calculation in 
  \rref{R_Calculate}.

\end{itemize}


\subsection{Nonfunctional Requirements}

\progname{} is small in size and relatively simple, so performance is not a 
priority. Any reasonable implementation will be very quick and use minimal 
storage. Rather than performance, the non-functional requirement priorities 
are correctness, understandability, reusability, maintainability, and 
portability. 

\begin{itemize}
\item[NF\refstepcounter{nfreqnum}\thenfreqnum \label{NF_Diag}:] \progname{} will
  generate diagrams with readable titles, axis labels, legends, and plot lines.
\end{itemize}

\section{Likely Changes}    

\noindent \begin{itemize}

\item[LC\refstepcounter{lcnum}\thelcnum\label{LC_SATP}:] Generation of diagrams 
  under non-SATP conditions [\aref{A_SATP}]
\item[LC\refstepcounter{lcnum}\thelcnum\label{LC_state}:] Consideration of gas 
  and solid equilibria [\aref{A_only_aq}]
\item[LC\refstepcounter{lcnum}\thelcnum\label{LC_act}:] Calculation using true 
  activity, with user supplied activity coefficients for more accurate 
  speciation [\aref{A_act_wat}, \aref{A_act_aq}] 

\end{itemize}


\section{Traceability Matrices and Graphs}

The purpose of the traceability matrices is to provide easy references on what
has to be additionally modified if a certain component is changed.  Every time a
component is changed, the items in the column of that component that are marked
with an ``X'' may have to be modified as well.  Table~\ref{Table:trace} shows 
the dependencies of theoretical models, general definitions, data definitions, 
and instance models with each other. Table~\ref{Table:R_trace} shows the
dependencies of instance models, requirements, and data constraints on each
other. Table~\ref{Table:A_trace} shows the dependencies of theoretical models,
general definitions, data definitions, instance models, and likely changes on
the assumptions.


\afterpage{
\begin{table}[h!]
\centering
\begin{tabular}{|c|c|c|c|c|c|c|}
\hline
	& \aref{A_pH}& \aref{A_SATP}& \aref{A_only_aq}& \aref{A_act_wat}& 
      \aref{A_act_aq}& \aref{A_rev}\\
\hline
\tref{T_MB}       & & & & & & \\ \hline
\tref{T_LMAF}     & & & & & & \\ \hline
\tref{T_LMAB}     & & & & & & \\ \hline
\tref{T_EQ}       & & & & & & \\ \hline
\tref{T_K}        & & & & & & \\ \hline
\dref{GD_water}   & &X& & & & \\ \hline
\ddref{DD_setK}   & & & & & &X\\ \hline
\ddref{DD_setM}   & & & & & & \\ \hline
\iref{IM_solve}   &X& &X&X&X& \\ \hline
\lcref{LC_SATP}   & &X& & & & \\ \hline
\lcref{LC_state}  & & &X& & & \\ \hline
\lcref{LC_act}    & & & &X&X& \\
\hline
\end{tabular}
\caption{Traceability Matrix Showing the Connections Between Assumptions and 
  Other Items}
\label{Table:A_trace}
\end{table}
}

\begin{table}[h!]
\centering
\begin{tabular}{|c|c|c|c|c|c|c|c|c|c|c|c|c|}
\hline        
	& \tref{T_MB}& \tref{T_LMAF}& \tref{T_LMAB} & 
      \tref{T_EQ}& \tref{T_K} & \dref{GD_water}& \ddref{DD_setK}& 
      \ddref{DD_setM}& \iref{IM_solve} \\
\hline
\tref{T_MB}      & & & & & & & & & \\ \hline
\tref{T_LMAF}    & & & & & & & & & \\ \hline
\tref{T_LMAB}    & & & & & & & & & \\ \hline
\tref{T_EQ}      & &X&X& & & & & & \\ \hline
\tref{T_K}       & & & &X& & & & & \\ \hline
\dref{GD_water}  & & & & &X& & & & \\ \hline
\ddref{DD_setK}  & & & & &X&X& & & \\ \hline
\ddref{DD_setM}  &X& & & & & & & & \\ \hline
\iref{IM_solve}  & & & & & & &X&X& \\
\hline
\end{tabular}
\caption{Traceability Matrix Showing the Connections Between TM, GD, DD, IM}
\label{Table:trace}
\end{table}

\begin{table}[h!]
\centering
\begin{tabular}{|c|c|c|}
\hline
	& \iref{IM_solve}& \sref{sec_DataConstraints} \\
\hline
\rref{R_Inputs}     & &\\ \hline
\rref{R_DerivedInputs}    & &\\ \hline
\rref{R_ValidateInputs}  & &X\\ \hline
\rref{R_Calculate}     &X &\\ \hline 
\rref{R_Output}       & & \\ 
\hline
\end{tabular}
\caption{Traceability Matrix Showing the Connections Between Requirements and 
  Instance Models}
\label{Table:R_trace}
\end{table}

\newpage
\newpage

\bibliographystyle {plainnat}
\bibliography {../../ReferenceMaterial/References}

~\newline

\wss{In your ReferenceMaterial folder, can you please put the initial document
  written by my brother?  We should also cite this document somewhere in your
  SRS.}

\end{document}