\documentclass[12pt, titlepage]{article}

\usepackage{amsmath, mathtools}

\usepackage[round]{natbib}
\usepackage{amsfonts}
\usepackage{amssymb}
\usepackage{graphicx}
\usepackage{colortbl}
\usepackage{xr}
\usepackage{hyperref}
\usepackage{longtable}
\usepackage{xfrac}
\usepackage{tabularx}
\usepackage{float}
\usepackage{siunitx}
\usepackage{booktabs}
\usepackage{multirow}
\usepackage[section]{placeins}
\usepackage{caption}
\usepackage{fullpage}

\hypersetup{
bookmarks=true,     % show bookmarks bar?
colorlinks=true,       % false: boxed links; true: colored links
linkcolor=red,          % color of internal links (change box color with linkbordercolor)
citecolor=blue,      % color of links to bibliography
filecolor=magenta,  % color of file links
urlcolor=cyan          % color of external links
}

\usepackage{array}

\input{../../Comments}

\newcommand\pfun{\mathrel{\ooalign{\hfil$\mapstochar\mkern5mu$\hfil\cr$\to$\cr}}}
\newcommand{\progname}{SpecGen}

\makeatletter
\newcommand*\bigcdot{\mathpalette\bigcdot@{.7}}
\newcommand*\bigcdot@[2]
  {\mathbin{\vcenter{\hbox{\scalebox{#2}{$\m@th#1\bullet$}}}}}
\makeatother

\begin{document}

\pagenumbering{gobble}

\title{CAS 741: Module Interface Specification\\[10pt]\Large Aqueous Speciation Diagram Generator}
\author{Steven Palmer\\\texttt{palmes4}}
\date{\today}
	
\maketitle

\pagenumbering{roman}

\setcounter{secnumdepth}{0}

\section{Revision History}

\begin{table}[hp]
\caption{Revision History} \label{TblRevisionHistory}
\begin{tabularx}{\textwidth}{llX}
\toprule
\textbf{Date} & \textbf{Developer(s)} & \textbf{Change}\\
\midrule
11.29.2017 & S. Palmer & Revision 0 submission\\
\bottomrule
\end{tabularx}
\end{table}

~\newpage

\section{Symbols, Abbreviations and Acronyms}

See \href{https://github.com/palmerst/cas741_sp/blob/master/Doc/SRS/SRS.pdf}{SRS Documentation}.

\newpage

\tableofcontents


\newpage


\pagenumbering{arabic}

\setcounter{secnumdepth}{3}

\section{Introduction}

The following document details the Module Interface Specifications for
the Aqueous Speciation Diagram Generator program. \wss{I suggest combining these
  two paragraphs into one paragraph.  Usually 1 sentence paragraphs aren't a
  great idea, and the ideas in the two paragraphs are related.}

Complementary documents include the System Requirement Specifications
and Module Guide.  The full documentation and implementation can be
found \href{https://github.com/palmerst/cas741_sp}{here}. 

\section{Notation}

The structure of the MIS for modules comes from \citet{HoffmanAndStrooper1995},
with the addition that template modules have been adapted from
\cite{GhezziEtAl2003}.  All template modules described in the MIS are assumed to
have getters for state variables; they will not be explicitly specified.
\wss{good idea} The mathematical notation comes from Chapter 3 of
\citet{HoffmanAndStrooper1995}, as well as \cite{GS1993}.  For instance, the
symbol := is used for a multiple assignment statement and conditional rules
follow the form
$(c_1 \Rightarrow r_1 | c_2 \Rightarrow r_2 | ... | c_n \Rightarrow r_n )$.

The following table summarizes the primitive data types used by \progname. 

\begin{center}
\renewcommand{\arraystretch}{1.2}
\noindent 
\begin{tabular}{l l p{7.5cm}} 
\toprule 
\textbf{Data Type} & \textbf{Notation} & \textbf{Description}\\ 
\midrule
character & char & a single symbol or digit\\
integer & $\mathbb{Z}$ & a number without a fractional component in (-$\infty$, $\infty$) \\
natural number & $\mathbb{N}$ & a number without a fractional component in [1, $\infty$) \\
real & $\mathbb{R}$ & any number in (-$\infty$, $\infty$)\\
\bottomrule
\end{tabular} 
\end{center}

\noindent
The specification of \progname \ uses some derived data types: sets, sequences, strings, and
tuples. Sets are an unordered unique collection of elements of the same data type.  
Sequences are lists filled with elements of the same data type. Strings
are sequences of characters. Tuples contain a list of values, potentially of
different types. In addition, \progname \ uses functions (partial and total), which
are defined by the data types of their inputs and outputs. Local functions are
described by giving their type signature followed by their specification.\\

\noindent
Since chemical reactions, species and elements are central to \progname{}, the following 
notation is introduced in an attempt to simplify the presentation of the MIS.\\

\noindent
Chemical reactions:\\

${\alpha}_1 {A}_{1} + \cdots + {\alpha}_n {A}_{n} 
    \rightleftharpoons {\beta}_1 {B}_{1} + \cdots + {\beta}_m {B}_{m} \qquad K$\\

where $A_i$ are reactant species, $\alpha_i$ are the stoichiometric numbers
of the reactant species, $n$ is the number of reactant species, $B_i$ are product species,
$\beta_i$ are the stoichiometric numbers of the product species, $m$ is the number of
product species, and $K$ is the equilibrium constant.\\

\noindent
Chemical species:\\

${{E}^1_{e1}}\cdots{{E}^n_{en}}^{z}_{(s)}$\\

where $E^i$ are chemical elements, $e_i$ are the number of each element $E^i$ in
the species, $n$ is the number of distinct elements in the species, $z$ is the
charge, and $s$ is the state.\\ \wss{What are the possible values of state?  It
  seems like state would be a type with 2 or 3 values.  I don't know if you
  cover a gaseous state, but I'm assuming that s could represent solid or
  liquid.  What is the notation for one versions the other?  I see from reading
  ahead that there are 4 possible states.  I didn't think of aqueous as a state,
  but that makes sense now.  You should define the State type here.}

\spc{I'm doing this so I don't have to do explicit parsing specification.  The specification for 
that was getting really ugly really fast and was too close to code.  Using ``pseudo-types'' 
(is that the right word?)
ends up specifying what the parsers will need to do, but doesn't lock in any particular input
string structure or parsing routine.  I think that's better anyway!}
\noindent
The following pseudo-types are used in this document.  The species and 
reaction pseudo-types are treated as strings with fields that correspond to the variables
in the notations given above.  Note that these types are only used for convenience with
respect to specification and
will not occur in the \progname{} implementation.

\begin{center}
\renewcommand{\arraystretch}{1.2}
\noindent 
\begin{tabular}{l l p{7.5cm}} 
\toprule 
\textbf{Pseudo Type} & \textbf{Notation} & \textbf{Resolves To}\\ 
\midrule
species & $\mathbb{S}$ & string representation of a chemical species\\
reaction & $\mathbb{X}$ & string representation of a chemical reaction \\
element & $\mathbb{E}$ & string representation of a chemical element \\
\bottomrule
\end{tabular} 
\end{center}

\wss{I've never heard of the term pseudo-types, but I like what you are doing
  here.  You have introduced a notation that makes the spec easier to read for a
  human being.  I think this is the right idea.  From the Drasil perspective,
  how we connect the human readable spec with the eventual computer code isn't
  clear, but from an MIS ``software engineering'' perspective, you are on the
  right track.}

\wss{Rather than $\mathbb{S}$ etc being strings, would it make more sense to
  define new types?  The type might be a tuple with fields for molecule and
  charge.  The molecule in turn could be a sequence  to tuples, where each of
  these tuples is an element and a number of occurrences.  There are only so
  many elements to make up the set of elements.  We could use something like
  this to turn your pseudo type into a real type.  We could go further and
  capture information about molecules so that the charge could be derived
  automatically (I think).  I'm not sure exactly what I'm thinking here, but it
  seems like an opportunity to capture information better than we would just
  carrying a string around.}

\section{Module Decomposition}

The following table is taken directly from the Module Guide document for this project.
\spc{Note to self:  update MG modules}
\begin{table}[h!]
\centering
\begin{tabular}{p{0.3\textwidth} p{0.6\textwidth}}
\toprule
\textbf{Level 1} & \textbf{Level 2}\\
\midrule

{Hardware-Hiding Module} & ~ \\
\midrule

\multirow{6}{0.3\textwidth}{Behaviour-Hiding Module} & External Interface\\
& Chemical System\\
& Species\\
& Reaction\\
& Input Conversion\\
& Calculation\\
\midrule

\multirow{2}{0.3\textwidth}{Software Decision} & Non-linear Solver\\
& Plotting\\
\bottomrule

\end{tabular}
\caption{Module Hierarchy}
\label{TblMH}
\end{table}

\newpage
~\newpage

\section{MIS of External Interface}

\subsection{Module}

SpecGen \wss{What does this module do that you couldn't get by just having the
  ChemSys module?  Except for the fact that ChemSys exports an ADT, the syntax
  of the interfaces are the same?}

\subsection{Uses}

ChemSys (Section \ref{Module:ChemSys})

\wss{Nice to see explicit cross referencing and hyperlinks in the uses section.}

\subsection{Syntax}

\subsubsection{Re-Exported Types}

ChemSys

\subsubsection{Re-Exported Access Programs}

ChemSys
\newline registerRxn
\newline registerTotal
\newline specGen


\subsection{Semantics}

N/A

~\newpage

\section{MIS of Chemical System} \label{Module:ChemSys}

\subsection{Template Module}

ChemSys

\subsection{Uses}

Species (Section \ref{Module:Species})
\newline ChemEq (Section \ref{Module:Reaction})
\newline Calculation (Section \ref{Module:Calculation})
\newline Plotting (Section \ref{Module:Plotting})

\subsection{Syntax}

\subsubsection{Exported Types}

ChemSys = ?

\subsubsection{Exported Access Programs}

\begin{center}
\begin{tabular}{p{3cm} p{3cm} p{3cm} p{3cm}}
\hline
\textbf{Name} & \textbf{In} & \textbf{Out} & \textbf{Exceptions} \\
\hline
ChemSys & - & ChemSys & - \\
registerRxn & $\mathbb{X}$ & - & - \\
registerTotal & $\mathbb{E}$, $\mathbb{R}$ & - & NonPositiveConc \\ 
specGen & fileName: String & - & - \\
\hline
\end{tabular}
\end{center}

\subsection{Semantics}

\subsubsection{State Variables}
rxns: Set of ChemEq
\newline species: $\mathbb{S} \pfun$ Species
\newline totals: $\mathbb{E} \pfun \mathbb{R}$ 

\subsubsection{Environment Variables}
FileSystem: The file system of the machine running \progname

\subsubsection{Access Routine Semantics}

\noindent ChemSys():
\begin{itemize}
\item transition: \begin{itemize} 
                  \item[] rxns := $\varnothing$
                  \item[] species := \{ \\
                  \hspace*{2em} (H$^+_{(aq)}$, Species(H$^+_{(aq)}$, aq, $1$, \{(H, 1)\})),\\
                  \hspace*{2em} (OH$^-_{(aq)}$, Species(OH$^-_{(aq)}$, aq, $-1$, \{(H, 1), (O, 1)\})),\\
                  \hspace*{2em} (H$_2$O$_{(l)}$, Species(H$_2$O$_{(l)}$, l, $0$, \{(H, 2), (O, 1)\})),\\
                  \} 
                  \item[] totals := $\varnothing$
                  \end{itemize}
\item output: \begin{itemize} 
              \item[] \emph{out} := self 
              \end{itemize}
\item exception: N/A
\end{itemize}

\noindent registerRxn(rx):
\begin{itemize}
\item transition: \begin{itemize} 
                  \item[] rxns := rxns $\cup$ ChemEq(\\
                  \hspace*{2em}log rx.K,\\
                  \hspace*{2em} $\{~\forall~i : \mathbb{N}~|~ i \leq \textrm{rx.n}  ~\bigcdot~ 
                    (\textrm{rx.A}_\textrm{i}, \textrm{rx.}\alpha_\textrm{i})~\}$,\\
                  \hspace*{2em} $\{~\forall~i : \mathbb{N}~|~ i \leq \textrm{rx.m}  ~\bigcdot~ 
                    (\textrm{rx.B}_i, \textrm{rx.}\beta_i)~\}$\\
                  )
                  \item[] species := species $\cup$ \{~$\forall~\textrm{sp} : \mathbb{S}
                    ~|~ \textrm{sp} \notin \textrm{dom(species)} \land \textrm{sp} \in \textrm{rx} ~\bigcdot~ $\\
                    \hspace*{2em}(sp, Species(sp, sp.s, sp.z, \{ $\forall~\textrm{i} : 
                      \mathbb{N}~|~ \textrm{i} \leq \textrm{sp.n} ~\bigcdot~ $(sp.E$_\textrm{i}$, sp.e$_\textrm{i}$)~\}))\\
                  \}
                  \end{itemize}
\item output: N/A
\item exception: N/A
\end{itemize}

\noindent registerTotal(elt, tot):
\begin{itemize}
\item transition: \begin{itemize} 
                  \item[] totals := totals $\cup$ \{~(elt, tot)~\}
                  \end{itemize}
\item output: N/A
\item exception: \begin{itemize}
                 \item[] \emph{exc} := (tot $\leq 0 \implies$ NonPositiveConc)
                 \end{itemize}
\end{itemize}

\noindent specGen(fileName):
\begin{itemize}
\item transition: \begin{itemize} 
                  \item[] FileSystem := plot(calcSpec(self), fileName)
                  \end{itemize}
\item output: N/A
\item exception: N/A
\end{itemize}

~\newpage

\section{MIS of Species} \label{Module:Species}

\subsection{Template Module}

Species

\subsection{Uses}

N/A

\subsection{Syntax}

\subsubsection{Exported Types}

Species = ?\\
State = \{s, l, g, aq\}

\subsubsection{Exported Access Programs}

\begin{center}
\begin{tabular}{p{2cm} p{4cm} p{4cm} p{2cm}}
\hline
\textbf{Name} & \textbf{In} & \textbf{Out} & \textbf{Exceptions} \\
\hline
Species & formula: $\mathbb{S}$ 
          \newline state: State
          \newline charge: $\mathbb{Z}$
          \newline composition: $\mathbb{E} \pfun \mathbb{N}$ 
                 & Species & - \\
\hline
\end{tabular}
\end{center}

\subsection{Semantics}

\subsubsection{State Variables}
formula:  $\mathbb{S}$
\newline state:  State
\newline charge:  $\mathbb{Z}$
\newline composition: $\mathbb{E} \pfun \mathbb{N}$

\wss{This feels close to a formal model of a species, except for the formula
  string and the string for E.  Could we also introduce a type for E?  There is
  a finite set of possibilities for this set.}

\subsubsection{Access Routine Semantics}

\noindent Species(formula, state, charge, composition):
\begin{itemize}
\item transition: \begin{itemize} 
                  \item[] self.formula := formula
                  \item[] self.state := state
                  \item[] self.charge := charge
                  \item[] self.composition := composition
                  \end{itemize}
\item output: \begin{itemize} 
              \item[] \emph{out} := self 
              \end{itemize}
\item exception: N/A
\end{itemize}


~\newpage


\section{MIS of Reaction} \label{Module:Reaction}

\subsection{Template Module}

ChemEq

\subsection{Uses}

N/A

\subsection{Syntax}

\subsubsection{Exported Types}

ChemEq = ?

\subsubsection{Exported Access Programs}

\begin{center}
\begin{tabular}{p{2cm} p{4cm} p{4cm} p{2cm}}
\hline
\textbf{Name} & \textbf{In} & \textbf{Out} & \textbf{Exceptions} \\
\hline
ChemEq & reaction: $\mathbb{X}$ 
         \newline logK: $\mathbb{R}$
         \newline r: $\mathbb{S} \pfun \mathbb{N}$
         \newline p: $\mathbb{S} \pfun \mathbb{N}$ 
                 & ChemEq & - \\
\hline
\end{tabular}
\end{center}

\subsection{Semantics}

\subsubsection{State Variables}
reaction: $\mathbb{X}$ 
\newline logK: $\mathbb{R}$
\newline r: $\mathbb{S} \pfun \mathbb{N}$
\newline p: $\mathbb{S} \pfun \mathbb{N}$ 


\subsubsection{Access Routine Semantics}

\noindent ChemEq(reaction, logK, r, p):
\begin{itemize}
\item transition: \begin{itemize} 
                  \item[] self.reaction := reaction
                  \item[] self.logK := logK
                  \item[] self.r := r
                  \item[] self.p := p
                  \end{itemize}
\item output: \begin{itemize} 
              \item[] \emph{out} := self 
              \end{itemize}
\item exception: N/A
\end{itemize}

~\newpage


\section{MIS of Input Conversion} \label{Module:Convert} 

\subsection{Module}

Convert

\subsection{Uses}

ChemSys (Section \ref{Module:ChemSys})
\newline Species (Section \ref{Module:Species})
\newline ChemEq (Section \ref{Module:Reaction})

\subsection{Syntax}

\subsubsection{Exported Access Programs}

\begin{center}
\begin{tabular}{p{3cm} p{3cm} p{3.2cm} p{2.8cm}}
\hline
\textbf{Name} & \textbf{In} & \textbf{Out} & \textbf{Exceptions} \\
\hline
makeRootFunc & syst: ChemSys & Set of \newline $\mathbb{R}^{n} \rightarrow \mathbb{R}$ & - \\
\hline
\end{tabular}
\end{center}

\subsection{Semantics}

\subsubsection{State Variables}
N/A


\subsubsection{Access Routine Semantics}

\noindent makeRootFunc(syst):
\begin{itemize}
\item transition: N/A
\item output: \begin{itemize} 
              \item[] \emph{out} := \{ $\forall$ r : ChemEq $|$ r $\in$ syst.rxn $\bigcdot$ \\
                  \hspace*{2em} $\lambda$($\forall$ s : $\mathbb{S}$ $|$ s $\in$ dom(syst.species) $\land$ s.s = aq $\bigcdot$ C$_s$) .\\
                  \hspace*{4em}$\sum$($\forall$ s : $\mathbb{S}$ $|$ s $\in$ dom(r.p) $\land$ s.s = aq $\bigcdot$ r.p(s) $\cdot$ log C$_s$)\\
                  \hspace*{4em}$- \sum$($\forall$ s : $\mathbb{S}$ $|$ s $\in$ dom(r.r) $\land$ s.s = aq $\bigcdot$ r.r(s) $\cdot$ log C$_s$)\\
                  \hspace*{4em}$-$ r.logK\\
              \} $\cup$ \{ $\forall$ e : $\mathbb{E}$ $|$ e $\in$ dom(syst.totals) $\bigcdot$ \\
                \hspace*{2em} $\lambda$($\forall$ s : $\mathbb{S}$ $|$ s $\in$ syst.species $\land$ s.s = aq  $\bigcdot$ C$_s$) .\\
                \hspace*{4em} log $\sum(\forall$ s : $\mathbb{S}$ $|$ s $\in$ dom(syst.species) \\
                \hspace*{6em} $\land$ e $\in$ dom(syst.species(s).composition) $\bigcdot$ \\
                \hspace*{8em} syst.species(s).composition(e) $\cdot$ C$_s$)\\
                \hspace*{4em} $-$ log syst.totals(e)\\
              \}
              \end{itemize}
\item exception: N/A
\end{itemize}


~\newpage


\section{MIS of Calculation} \label{Module:Calculation} 

\subsection{Module}

Calculation

\subsection{Uses}

ChemSys (Section \ref{Module:ChemSys})
\newline Convert (Section \ref{Module:Convert})
\newline Solver (Section \ref{Module:NLS})

\subsection{Syntax}

\subsubsection{Exported Access Programs}

\begin{center}
\begin{tabular}{p{2cm} p{3cm} p{4.5cm} p{2.5cm}}
\hline
\textbf{Name} & \textbf{In} & \textbf{Out} & \textbf{Exceptions} \\
\hline
calcSpec & syst: ChemSys & Set of ($\mathbb{R}$, $\mathbb{R}^{n}$) & - \\
\hline
\end{tabular}
\end{center}

\subsection{Semantics}

\subsubsection{State Variables}
N/A


\subsubsection{Access Routine Semantics}

\noindent calcSpec(syst):
\begin{itemize}
\item transition: N/A
\item output: \begin{itemize} 
              \item[] \emph{out} := \{ $\forall$ pH : $\mathbb{R}$ $|$ 0 $\leq$ pH $\leq$ 14 $\bigcdot$\\
                \hspace*{2em}(pH, deLog(rootSolve(partialApply(makeRootFunc(syst), pH))))\\
                \}
              \end{itemize}
\item exception: N/A
\end{itemize}

\subsubsection{Local Functions}
partialApply: (Set of $\mathbb{R}^{n} \rightarrow \mathbb{R}$) $\times$ $\mathbb{R}$ $\rightarrow$ 
  (Set of $\mathbb{R}^{n - 2} \rightarrow \mathbb{R}$)
\newline partialApply(eqs, pH) $\equiv$ \{ $\forall$ f : eqs $\bigcdot$ f(C$_{H^+}$ = 10$^{-pH}$, C$_{OH^-}$ = 10$^{pH - 14}$) \}\\

\noindent deLog:  $\mathbb{R}^{n} \rightarrow \mathbb{R}^{n}$
\newline deLog(concs) $\equiv$ map($\lambda$x.$10^x$, concs)

~\newpage


\section{MIS of Non-Linear Solver} \label{Module:NLS} 

\subsection{Module}

Solver

\subsection{Uses}

N/A

\subsection{Syntax}

\subsubsection{Exported Access Programs}

\begin{center}
\begin{tabular}{p{2.8cm} p{3.6cm} p{2.8cm} p{2.8cm}}
\hline
\textbf{Name} & \textbf{In} & \textbf{Out} & \textbf{Exceptions} \\
\hline
rootSolve & eqs: Set of $\mathbb{R}^n \rightarrow \mathbb{R}$ \newline for any n : $\mathbb{N}$ & $\mathbb{R}^n$ & - \\
\hline
\end{tabular}
\end{center}

\subsection{Semantics}

\subsubsection{State Variables}
N/A


\subsubsection{Access Routine Semantics}

\noindent rootSolve(eqs):
\begin{itemize}
\item transition: N/A
\item output: \begin{itemize} 
              \item[] \emph{out} := res, where ($\forall$ f : eqs $\bigcdot$ f(res) = 0)
              \end{itemize}
\item exception: N/A
\end{itemize}


~\newpage

\section{MIS of Plotting} \label{Module:Plotting} 

\subsection{Module}

Plotting

\subsection{Uses}

N/A

\subsection{Syntax}

\subsubsection{Exported Access Programs}

\begin{center}
\begin{tabular}{p{2cm} p{4cm} p{3cm} p{3cm}}
\hline
\textbf{Name} & \textbf{In} & \textbf{Out} & \textbf{Exceptions} \\
\hline
plot & data: Set of ($\mathbb{R}$, $\mathbb{R}^n$) 
  \newline \hspace*{1em}for any n : $\mathbb{N}$
  \newline fileName: String & - & - \\
\hline
\end{tabular}
\end{center}

\subsection{Semantics}

\subsubsection{State Variables}
N/A

\subsubsection{Environment Variables}
FileSystem: The file system of the machine running \progname

\subsubsection{Access Routine Semantics}

\noindent plot(data, fileName):
\begin{itemize}
\item transition: Creates a new image file with path \emph{fileName} in 
  \emph{FileSystem} with a 2D plot of the set of data in \emph{data}.  The 
  first item in each tuple of \emph{data} is the x-value and the second item 
  contains y-values that should be paired with that x-value.    
\item output: N/A
\item exception: N/A
\end{itemize}

\newpage

\wss{Great MIS.  If we can find a way to capture the species information in a
  type instead of a pseudo type, I think we will be getting closer to something
  that Drasil can use.}

\bibliographystyle {plainnat}
\bibliography {../../../ReferenceMaterial/References}

\end{document}