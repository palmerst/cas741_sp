\documentclass[12pt, titlepage]{article}

\usepackage{booktabs}
\usepackage{tabularx}
\usepackage{hyperref}
\hypersetup{
    colorlinks,
    citecolor=black,
    filecolor=black,
    linkcolor=red,
    urlcolor=blue
}
\usepackage[round]{natbib}
\usepackage{fullpage}
\usepackage{scrextend}
\usepackage{amsmath}

\newcommand{\progname}{SpecGen}

\newcounter{testnum} %Instance Number
\newcommand{\tthetestnum}{T\thetestnum}
\newcommand{\testref}[1]{T\ref{#1}}
\newcommand{\sref}[1]{\S~\ref{#1}}

\input{../Comments}

\begin{document}
\pagenumbering{gobble}

\title{CAS 741: Test Plan\\[10pt]\Large Aqueous Speciation Diagram Generator}
\author{Steven Palmer\\\texttt{palmes4}}
\date{\today}
	
\maketitle

\pagenumbering{roman}

\setcounter{secnumdepth}{0}

\section{Revision History}

\begin{table}[hp]
\caption{Revision History} \label{TblRevisionHistory}
\begin{tabularx}{\textwidth}{llX}
\toprule
\textbf{Date} & \textbf{Developer(s)} & \textbf{Change}\\
\midrule
10.13.2017 & S. Palmer & First revision of document\\
10.25.2017 & S. Palmer & Revision 0 submission\\
\bottomrule
\end{tabularx}
\end{table}

~\newpage


\section{Symbols, Abbreviations and Acronyms}

\renewcommand{\arraystretch}{1.2}
\begin{tabular}{l l} 
  \toprule		
  \textbf{symbol} & \textbf{description}\\
  \midrule
  \progname{} & The Aqueous Speciation Diagram Generator program\\  
  SRS & Software Requirements Specification\\
  T & Test\\
  \bottomrule
\end{tabular}\\

\newpage

\tableofcontents


\newpage


\pagenumbering{arabic}

\setcounter{secnumdepth}{3}

\section{General Information}

This document provides a detailed description of the testing that will be 
carried out on the Aqueous Speciation Diagram Generator program (herein referred 
to as \progname{}).  

\subsection{Purpose}
The purpose of this document is to provide a comprehensive plan for testing the 
\progname{} software against the requirements described in the
\progname{} SRS.  \wss{An explicit web-link to your GitHub repo would be nice.}

\subsection{Scope}
The test plan is narrowed to the following scope:
\begin{itemize}
\item The tests outlined in this document are limited to the verification of the 
  requirements given in the \progname{} SRS.  The validation of the requirements 
  will be carried out via correspondance with Dr.\ Scott Smith.
  \wss{Add information on Dr.\ Smith's affiliation}
\item The tests outlined in this document are limited to dynamic tests only.  
  Due to the small size and low complexity of the \progname{} program, no formal
  static testing (code walkthroughs, code inspections, etc.) will be carried out.
\item The \progname{} software will be written in Python.  The testing of 
  implementations in other languages will not be considered in this document.
\end{itemize}

\section{Plan}
\label{SecPlan}
	
\subsection{Software Description}
Chemical speciation refers to the stable (equilibrium) distribution of chemical 
species in a given chemical system.   Speciation diagrams, which plot species 
concentrations against an independently varied parameter of the system, are 
useful tools for displaying speciation data in a concise and easy to use format.  

\progname{} will produce a speciation diagram given a set of chemical reactions, 
equilibrium constants, and element totals that define a chemical system.  
\progname{} will be specific to speciation of ions in aqueous systems under 
varying pH, which is of particular importance in the fields of aqueous process 
engineering and hydrometallurgy.  The diagram generated by \progname{} will plot 
speciation of all aqueous species (excluding H$^+$ and OH$^-$) across the pH 
range 0 to 14.

\subsection{Test Team}
The test team includes the following members:
\begin{itemize}
\item Steven Palmer
\end{itemize}

\subsection{Automated Testing Approach}
The automated testing for \progname{} will be carried out using a set of unit 
and integrations tests.  A test coverage analysis of these tests will be 
carried out to ensure that testing is as complete as possible.  The target 
for this analysis is 100\% statement coverage. 

Regression testing will be used during the implementation stage and for any 
future changes.  Since \progname{} is small in scope and will be implemented 
by a single developer, other forms of automated testing, such as continuous 
integration testing, will not be considered.


\subsection{Verification Tools}
The following tools will be used to facilitate testing:

\begin{enumerate}
\item {\bf PyUnit} (a unit testing framework for Python) will be used to write 
  and run unit tests
\item {\bf Coverage.py} will be used to assess test coverage
\item {\bf make} will be used to automate the building and execution of the test 
  program
\end{enumerate}



\subsection{Non-Testing Based Verification}
N/A

\section{System Test Description}
\spc{Intial states, inputs, outputs will be formalized/revised as implementation 
  proceeds.}	
\subsection{Tests for Functional Requirements}

\subsubsection{Input Testing}

\noindent {\bf T\refstepcounter{testnum}\thetestnum \label{T_In_EPG_I}: Equation 
Parser, Integration}\\
\begin{addmargin}[2em]{0em}
\begin{description}
\item[Type:] Functional, Automatic, Integration
					
\item[Initial State:] N/A
					
\item[Input:] Input file with general chemical reaction\\
\spc{TODO: change this to the actual input file text with implementation}
\wss{I know you cannot give the file at this point, since this
  document was written before the implementation, but more information
  would be nice.  What is the sample equation you are going to parse?
  Couldn't you write it here in symbolic form?  Could you write
  something like what you wrote for T3?}

\item[Output:] Data structure with input reaction equation components\\
\spc{TODO: change this to the actual output structure when implemented}
					
\item[How test will be performed:] Automated integration test \\
\end{description}
\end{addmargin}	

\noindent {\bf T\refstepcounter{testnum}\thetestnum \label{T_In_IDS_I}: Input 
Data Structure, Integration}\\
\begin{addmargin}[2em]{0em}
\begin{description}
\item[Type:] Functional, Automatic, Integration
					
\item[Initial State:] N/A
					
\item[Input:] Input file\\
\spc{TODO: change this to the actual input file text with implementation}

\item[Output:] Input parameters stored by the program match the input file\\
\spc{TODO: change this to the actual output structure when implemented}
					
\item[How test will be performed:] Automated integration test\\
\end{description}
\end{addmargin}		


\subsubsection{Derived Input Testing}

\begin{addmargin}[2em]{0em}
\noindent {\bf T\refstepcounter{testnum}\thetestnum \label{T_DIn_SL_I}: Species 
List from Equation, Distinct Species in Equation}\\
\begin{addmargin}[2em]{0em}
\begin{description}
\item[Type:] Functional, Automatic, Integration
					
\item[Initial State:] N/A
					
\item[Input:] Input file with reaction:\\
H$_2$CO$_3$$_\text{(aq)}$ $\rightleftharpoons$ H$^+$ + HCO$_3^-$\\
\spc{TODO: change this to 
  the actual input file text with implementation}
					
\item[Output:] List of species [H$_2$O, H$^+$, OH$^-$, H$_2$CO$_3$$_\text{(aq)}$, HCO$_3^-$]
					
\item[How test will be performed:] Automated integration test\\
\end{description}
\end{addmargin}

\noindent {\bf T\refstepcounter{testnum}\thetestnum \label{T_DIn_SL_R_I}: Species 
List from Equations, Repeated Species in Equations}\\
\begin{addmargin}[2em]{0em}
\begin{description}
\item[Type:] Functional, Automatic, Integration
					
\item[Initial State:] N/A
					
\item[Input:] Input file with reactions:\\
H$_2$CO$_3$$_\text{(aq)}$ $\rightleftharpoons$ H$^+$ + HCO$_3^-$\\
HCO$_3^-$ $\rightleftharpoons$ H$^+$ + CO$_3^{2-}$\\
\spc{TODO: change this to 
  the actual input file text with implementation}
					
\item[Output:] List of species [H$_2$O, H$^+$, OH$^-$, H$_2$CO$_3$$_\text{(aq)}$, HCO$_3^-$, CO$_3^{2-}$]
					
\item[How test will be performed:] Automated integration test\\
\end{description}
\end{addmargin}
\end{addmargin}

\subsubsection{Input Constraint Testing}

\wss{Although it might be hard to maintain without something like
  Drasil, it would be nice to see a pointer to the Table in the SRS
  where the input constraints are listed.}

\begin{addmargin}[2em]{0em}
\noindent {\bf T\refstepcounter{testnum}\thetestnum \label{T_CIn_EC_Neg_I}: Element 
Total Concentration, Integration}\\
\begin{addmargin}[2em]{0em}
\begin{description}
\item[Type:] Functional, Automatic, Integration
					
\item[Initial State:] N/A
					
\item[Input:] Input file with Fe$_{\text{tot}}$ = -1.0\\
\spc{TODO: change this to the actual input file text with implementation}
					
\item[Output:] Error
					
\item[How test will be performed:] Automated integration test\\
\end{description}
\end{addmargin}

\noindent {\bf T\refstepcounter{testnum}\thetestnum \label{T_CIn_EC_Zero_I}: Element 
Total Concentration, Zero, Integration}\\
\begin{addmargin}[2em]{0em}
\begin{description}
\item[Type:] Functional, Automatic, Integration
					
\item[Initial State:] N/A
					
\item[Input:] Input file with Fe$_{\text{tot}}$ = 0.0\\
\spc{TODO: change this to the actual input file text with implementation}
					
\item[Output:] Error
					
\item[How test will be performed:] Automated integration test\\
\end{description}
\end{addmargin}

\noindent {\bf T\refstepcounter{testnum}\thetestnum \label{T_CIn_EC_Pos_I}: Element 
Total Concentration, Positive, Integration}\\
\begin{addmargin}[2em]{0em}
\begin{description}
\item[Type:] Functional, Automatic, Integration
					
\item[Initial State:] N/A
					
\item[Input:] Input file with Fe$_{\text{tot}}$ = 1.0\\
\spc{TODO: change this to the actual input file text with implementation}
					
\item[Output:] Fe$_{\text{tot}}$ = 1.0
					
\item[How test will be performed:] Automated integration test\\
\end{description}
\end{addmargin}
\end{addmargin}



\subsubsection{Calculation Testing}

\begin{addmargin}[2em]{0em}
\noindent {\bf T\refstepcounter{testnum}\thetestnum \label{T_Calc_Solver_Parallel_I}: 
Non-Linear Equation System Solution, Integration}\\
\begin{addmargin}[2em]{0em}
\begin{description}
\item[Type:] Functional, Automatic, Parallel, Integration
					
\item[Initial State:] N/A
					
\item[Input:] Input file with same data as the MATLAB implementation\\
\spc{TODO: change this to the actual input file text with implementation}
					
\item[Output:] Solution that matches the MATLAB implementation\\
\spc{TODO: get data from MATLAB implementation}
					
\item[How test will be performed:] Automated integration test\\
\end{description}
\end{addmargin}

\wss{You could easily be more specific here.  What input file for
  Matlab?  Probably the data from Dr.\ Smith's original problem
  description.}

\wss{You will not get exact agreement with the output of the Matlab
  implementation.  You should explain that the output will include a
  summary statistic for the relative difference between the two
  solutions.  My initial thought would be a separate statistic for
  each species ($\mathbf{x}$), something like 
$$\frac{|| \mathbf{x}_{\text{Mat}} -
  \mathbf{x}_{\text{\progname{}}}||} {||\mathbf{x}_{\text{Mat}}||}$$
where the vertical bars represent a vector norm, like the Euclidean
norm, or the infinity norm.
}
\wss{This comment about not getting exact agreement with the expected
  output applies elsewhere in the document as well.}

\noindent {\bf T\refstepcounter{testnum}\thetestnum \label{T_Calc_Solver_I}: 
Non-Linear Equation System Solution, Integration}\\
\begin{addmargin}[2em]{0em}
\begin{description}
\item[Type:] Functional, Automatic, Integration
					
\item[Initial State:] N/A
					
\item[Input:] Input file with redundant equation:\\
Fe$^{3+}$ $\rightleftharpoons$ Fe$^{3+}$, K = 1.0\\
Fe$_{\text{tot}}$ = 0.1\\
\spc{TODO: change this to the actual input file text with implementation}
					
\item[Output:] [Fe$^{3+}$] = 0.1 M at all pH
					
\item[How test will be performed:] Automated integration test\\
\end{description}
\end{addmargin}
\end{addmargin}

\subsubsection{Output Testing}

\begin{addmargin}[2em]{0em}
\noindent {\bf T\refstepcounter{testnum}\thetestnum \label{T_Out_SpecGen_I}: 
Speciation Diagram, Integration}\\
\begin{addmargin}[2em]{0em}
\begin{description}
\item[Type:] Functional, Manual, Integration
					
\item[Initial State:] N/A
					
\item[Input:] Input file with:\\
CO$_2$$_\text{(aq)}$ + H$_2$O $\rightleftharpoons$ H$_2$CO$_3$$_\text{(aq)}$, 
  K = $1.7\times10^{−3}$\\
H$_2$CO$_3$$_\text{(aq)}$ $\rightleftharpoons$ H$^+$ + HCO$_3^-$, 
  K = $2.5\times10^{−4}$\\
HCO$_3^-$ $\rightleftharpoons$ H$^+$ + CO$_3^{2-}$, K = $4.69\times10^{−11}$\\
C$_{\text{tot}}$ = 0.1\\
\spc{TODO: change this to the actual input file text with implementation}
					
\item[Output:] Correct carbonate speciation diagram\\
\spc{TODO: should I link to diagram here?}
					
\item[How test will be performed:] The generated speciation diagram will be 
manually reviewed to ensure that it matches the diagram expected by the input 
specifications.\\
\end{description}
\end{addmargin}
\end{addmargin}


\subsection{Tests for Nonfunctional Requirements}

\subsubsection{Output Readability Testing}
		
\noindent {\bf }
\begin{addmargin}[2em]{0em}
\noindent {\bf T\refstepcounter{testnum}\thetestnum \label{T_NF_Read}: Readability of 
generated speciation diagram}\\
\begin{addmargin}[2em]{0em}
\begin{description}
\item[Type:] Nonfunctional, Manual
					
\item[How test will be performed:] This is a qualitative test to ensure that the 
diagrams generated by \progname{} are readable (axis labels visible, curves 
distinguishable from eachother, \wss{proof read} legend/labelling of curves, etc.).\\
\end{description}
\end{addmargin}

\end{addmargin}

\wss{It wouldn't be bad to add an acceptance test where you explicitly
  plan to ask Dr.\ Smith to review the program.  This would implicitly
  capture functional and nonfunctional requirements, to some extent.}

\section{Traceability Between System Tests and Requirements}
A trace between system tests and requirements is provided in 
\hyperref[tab:reqtrace]{Table~\ref*{tab:reqtrace}}.

\begin{table}[h]
\caption{Requirements Traceability} \label{tab:reqtrace}
\centering
\begin{tabularx}{0.55\textwidth}{p{4cm}X}
\toprule {\bf Requirement} & {\bf Test(s)}\\
\midrule
R1	&	\testref{T_In_EPG_I}, \testref{T_In_IDS_I}\\
R2	&	\testref{T_DIn_SL_I}, \testref{T_DIn_SL_R_I}\\
R3	&	\testref{T_CIn_EC_Neg_I}, \testref{T_CIn_EC_Zero_I}, \testref{T_CIn_EC_Pos_I}\\
R4	&	\testref{T_Calc_Solver_Parallel_I}, \testref{T_Calc_Solver_I}\\
R5	&	\testref{T_Out_SpecGen_I}\\
NF1 & \testref{T_NF_Read}~\spc{TODO: add NF1 to SRS}\\
\bottomrule
\end{tabularx}
\end{table}

\newpage		
\section{Unit Testing Plan}
\spc{Just a rough idea of unit tests -- will update with implementation.}
		
\subsection{Input Module Testing}

\begin{addmargin}[2em]{0em}
\noindent {\bf T\refstepcounter{testnum}\thetestnum \label{T_In_EPG}: Equation 
  Parser, General}\\
\begin{addmargin}[2em]{0em}
\begin{description}
\item[Type:] Functional, Automatic, Unit
					
\item[Initial State:] ?
					
\item[Input:] String representation of chemical reaction
					
\item[Output:] Data structure of reaction equation components
					
\item[How test will be performed:] Automated unit test \\
\end{description}
\end{addmargin}			
\noindent {\bf T\refstepcounter{testnum}\thetestnum \label{T_In_EPE1}: Equation 
  Parser, Edge Case 1}\\
\begin{addmargin}[2em]{0em}
\begin{description}
\item[Type:] Functional, Automatic, Unit
					
\item[Initial State:] ?
					
\item[Input:] String representation of chemical equation that hits edges cases 
  (want full coverage)
					
\item[Output:] Data structure of reaction equation components
					
\item[How test will be performed:] Automated unit test\\
\end{description}
\end{addmargin}
\noindent \spc{Edge cases will be added during implementation (after data 
structures and exact format of input is decided).  For simplicity going forward, 
edge cases are ignored (since at this point I don't know the number/nature of 
the edge cases).  They will be added as the implementation is developed.}\\

\noindent {\bf T\refstepcounter{testnum}\thetestnum \label{T_In_IDS}: Input Data 
Structure}\\
\begin{addmargin}[2em]{0em}
\begin{description}
\item[Type:] Functional, Automatic, Unit
					
\item[Initial State:] ?
					
\item[Input:] Input data (?)
					
\item[Output:] Data structure works -- probably several tests
					
\item[How test will be performed:] Automated unit test\\
\end{description}
\end{addmargin}
\end{addmargin}


\subsection{Derived Input Module Testing}

\begin{addmargin}[2em]{0em}
\noindent {\bf T\refstepcounter{testnum}\thetestnum \label{T_DIn_SL}: Species 
List from Equation}\\
\begin{addmargin}[2em]{0em}
\begin{description}
\item[Type:] Functional, Automatic, Unit
					
\item[Initial State:] ?
					
\item[Input:] String representation of chemical reaction
					
\item[Output:] List of species contained in reaction
					
\item[How test will be performed:] Automated unit test\\
\end{description}
\end{addmargin}
\end{addmargin}

\subsection{Input Constraints Module Testing}

\begin{addmargin}[2em]{0em}
\noindent {\bf T\refstepcounter{testnum}\thetestnum \label{T_CIn_ECNeg}: 
Negative Element Total Concentration}\\
\begin{addmargin}[2em]{0em}
\begin{description}
\item[Type:] Functional, Automatic, Unit
					
\item[Initial State:] ?
					
\item[Input:] Negative concentration
					
\item[Output:] Error
					
\item[How test will be performed:] Automated unit test\\
\end{description}
\end{addmargin}

\noindent {\bf T\refstepcounter{testnum}\thetestnum \label{T_CIn_ECZero}: Zero 
Element Total Concentration}\\
\begin{addmargin}[2em]{0em}
\begin{description}
\item[Type:] Functional, Automatic, Unit
					
\item[Initial State:] ?
					
\item[Input:] Zero concentration
					
\item[Output:] Error
					
\item[How test will be performed:] Automated unit test\\
\end{description}
\end{addmargin}

\noindent {\bf T\refstepcounter{testnum}\thetestnum \label{T_CIn_ECPos}: 
Positive Element Total Concentration}\\
\begin{addmargin}[2em]{0em}
\begin{description}
\item[Type:] Functional, Automatic, Unit
					
\item[Initial State:] ?
					
\item[Input:] Positive concentration
					
\item[Output:] Positive concentration
					
\item[How test will be performed:] Automated unit test\\
\end{description}
\end{addmargin}
\end{addmargin}

\subsubsection{Conversion Module Testing}
\spc{\progname{} will convert inputs into a set of non-linear equations to be 
solved by a non-linear solver.  This will involve some sort of module/code to 
be tested here.}

\subsection{Calculation Module Testing}
\begin{addmargin}[2em]{0em}
\noindent {\bf T\refstepcounter{testnum}\thetestnum \label{T_Calc_Solver}: 
Non-Linear Equation System Solution}\\
\begin{addmargin}[2em]{0em}
\begin{description}
\item[Type:] Functional, Automatic, Unit
					
\item[Initial State:] ?
					
\item[Input:] System of non-linear equations
					
\item[Output:] Solution that satisfies original system of equations
					
\item[How test will be performed:] Automated unit test\\
\end{description}
\end{addmargin}
\end{addmargin}

\subsection{Output Module Testing}

\begin{addmargin}[2em]{0em}
\noindent {\bf T\refstepcounter{testnum}\thetestnum \label{T_Out_SpecGen}: 
Speciation Diagram}\\
\begin{addmargin}[2em]{0em}
\begin{description}
\item[Type:] Functional, Manual, Unit
					
\item[Initial State:] ?
					
\item[Input:] Output specifications
					
\item[Output:] Speciation diagram
					
\item[How test will be performed:] The generated speciation diagram will be 
manually reviewed to ensure that it matches the specified output.\\
\end{description}
\end{addmargin}
\end{addmargin}

\newpage
\section{Traceability Between Unit Tests and Modules}
A trace between unit tests and modules is provided in 
\hyperref[tab:modtrace]{Table~\ref*{tab:modtrace}}.

\begin{table}[h]
\caption{Module Traceability} \label{tab:modtrace}
\centering
\begin{tabularx}{0.55\textwidth}{p{4cm}X}
\toprule {\bf Module} & {\bf Test(s)}\\
\midrule
\spc{TODO:  fill in once implementation is done}&\\
\bottomrule
\end{tabularx}
\end{table}

%\bibliographystyle{plainnat}

%\bibliography {../../ReferenceMaterial/References}

\end{document}