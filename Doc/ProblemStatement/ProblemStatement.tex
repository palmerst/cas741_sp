\documentclass{article}

\usepackage{tabularx}
\usepackage{booktabs}
\usepackage[explicit]{titlesec}
\usepackage{fullpage}
\usepackage{titling}

\setlength{\droptitle}{-14ex}  

\titleformat{\section}
  {\normalfont\large\bfseries}{\thesection}{1em}{{#1}.}

\titleformat{\subsection}
  {\normalfont\bfseries}{\thesection}{1em}{{#1}.}



\title{CAS 741: Problem Statement\\[10pt]\Large Aqueous Speciation Diagram Generator}

\author{Steven Palmer\\\texttt{palmes4}}

\date{}

\input{../Comments}

\begin{document}

\pagenumbering{gobble}

\maketitle

\begin{table}[hp]
\caption{Revision History} \label{TblRevisionHistory}
\begin{tabularx}{\textwidth}{llX}
\toprule
\textbf{Date} & \textbf{Developer(s)} & \textbf{Change}\\
\midrule
9.15.2017 & S. Palmer & First revision of document\\
\bottomrule
\end{tabularx}
\end{table}

\section*{Problem}
Chemical speciation refers to the stable (equilibrium) distribution of chemical species in a given chemical system.   Speciation diagrams, which plot species concentrations against an independently varied parameter of the system, are useful tools for displaying speciation data in a concise and easy to use format.  The production of speciation diagrams requires solving a set of non-linear equations which arise from the chemical reactions of the species present in a chemical system.  As the number of reactions taking place in a system increases, producing speciation diagrams can quickly become a tedious undertaking when done manually.

\section*{Proposed Solution}
The proposed software solution will produce a speciation diagram given a set of chemical reactions and element totals that define a chemical system as inputs \sp{I decided to start with a generalized version rather than specific to Fe(OH)$_3$}.  The software will be specific to speciation of ions in aqueous systems under varying pH, which is of particular importance in the fields of aqueous process engineering and hydrometallurgy.  The diagram generated by the software will plot speciation of all aqueous species (excluding H$^+$ and OH$^-$) across the pH range 0 to 14.

\section*{Context}
\subsection*{Environment}
The software will be compatible with Windows 10, OSX ? \sp{What version of OSX are you running?}, and Ubuntu Linux 17.04.  Compatibility with other versions of Windows, OSX and Linux is likely, but will not be guaranteed nor tested.

\subsection*{Stakeholders}
Specific stakeholders include:
\begin{itemize}
\item Steven Palmer
\item Dr. Spencer Smith
\item Dr. Scott Smith
\item Students of CAS 741 
\item Individuals studying or working in fields related to chemistry
\end{itemize}

\end{document}